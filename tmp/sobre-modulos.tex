\item As funções \elem\ e \delete\ estão definidas no módulo
  \Prelude\ de Haskell, que é importado por todo módulo Haskell; foram
  definidos apenas por questões de completeza e clareza (de fato, o
  módulo que define \elem\ e \delete\ teria que impedir a importação
  dessas funções definidas em \Prelude, com {\tt
    \import\ \Prelude\ \hiding (\elem, \delete)}.

  Um programa em Haskelll é uma sequência de módulos, sendo que um
  deles, o módulo principal, deve conter uma função de nome \main, na
  qual a execução do programa inicia.

  A definição de um módulo com nome (digamos) $A$ que importa um
  módulo {\it MB\/} (abreviado para $B$) que esconde (não importa) de
  $B$ os nomes \elem\ e \delete\ pode ser feita como a seguir: 

  \prog{\module\ $A$ (nomes-exportados) \where\ \\
             \import\ {\it MB\/} [\as\ $B$] [[hiding] (\elem, \delete)]\\
             definições-de-funções 
       }

  Se {\tt (nomes-exportados)} não for especificado, todos os nomes no
  escopo global são exportados (para não exportar nennum nome use {\tt
    ()}). Vários cláusulas de importação podem ser usados (um para
  cada módulo importado). Os nomes a serem importados podem ser
  indicados (usando uma cláusula de importação que não usa a
  palavra-chave \hiding), do contrário todos os nomes do módulo
  importado se tornam visíveis.
