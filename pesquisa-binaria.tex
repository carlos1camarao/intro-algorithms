% !TEX encoding = ISO-8859-1
\subsection{Pesquisa bin�ria}
\label{pesquisa-binaria}

Pesquisa bin�ria � apresentada a seguir usando �rvores de pesquisa
bin�ria, em Haskell, e arranjos ordenados, em pseudo-c�digo
imperativo.

Uma �rvore de pesquisa bin�ria � uma �rvore bin�ria --- isto �, um
�rvore com duas sub-�rvores (possivelmente vazias), digamos, �
esquerda e � direita --- com os elementos contidos nos nodos e tal
que: todo elemento contido em um nodo � maior que os elementos
contidos na sub-�rvore � esquerda e menor que os elementos contidos na
sub-�rvore � direita.

Dois exemplos de �rvores bin�rias de pesquisa com os elementos de 1 a
7 s�o mostradas abaixo. 

\begin{verbatim}
  4                    4
 / \                  / \
1   5                2   6
 \   \              / \  /\
 3    6            1  3 5  7 
/      \ 
2       7
\end{verbatim}

A propriedade fundamental de uma �rvore bin�ria de pesquisa � o acesso
eficiente (como vamos ver, logar�tmico) a um elemento.

\subsubsection{Vers�o funcional}





