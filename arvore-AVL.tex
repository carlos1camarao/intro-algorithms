\section{Árvore AVL}
\label{sec:arvore-AVL}

O nome AVL é proveniente das iniciais dos sobrenomes dos dois
pesquisadores russos G.~M.~\underline{A}delson-\underline{V}elsky e
E.~M.~\underline{L}andis, que foram os primeiros a definir e realizar
trabalhos com esse tipo de árvore.

Seja $n$ um nó de uma árvore binária, $ad_n$ e $ae_n$ as alturas da
sub-árvore esquerda e direita de $n$, respectivamente, e seja $k_n =
ae_n - ad_n $ o {\em fator de balanceamento\/} do nodo (i.e.~o fator
de balanceamento é igual ao valor da diferença entre as alturas de
suas sub-árvores).

Uma árvore AVL é uma árvore de pesquisa binária na qual $\delta_n$ é
igual a 0 ou 1 ou -1, para todo nodo $n$.

Por exemplo, a ávore binária de pesquisa abaixo à esquerda é uma
árvore AVL, enquanto a da direita não é. 

\begin{verbatim}
      5                 5
     / \               /
    2   6             2 
   /                 / 
  1                 1   
\end{verbatim}

O algoritmo de pesquisa em uma árvore AVL é o mesmo do algoritmo de
pesquisa em uma árvore binária de pesquisa. 

Os algoritmos de inserção e remoção são apresentados nas subseções
seguintes.

\subsection{Inserção}
\label{sec:insercao-em-arvores-AVL-versao-func}

Após inserção de nodo em uma árvore AVL, é feita uma verificação do
fator de balanceamento de cada nodo que está no caminho da raiz até o
nodo inserido. Se a inserção tornar o fator de balanceamento maior que
1 ou menor que -1, a sub-árvore com raiz nesse nodo é rebalanceada,
por meio de uma {\em rotação}, para que a condição-AVL volte a ser
satisfeita. Como a inserção de elemento em uma árvore pode aumentar a
altura da árvore em no máximo 1, o fator de balanceamento deve ser,
logo após a inserção de nodo na sub-árvore esquerda e antes do
rebalanceamento, no máximo igual a 2; e, logo após a inserção de nodo
na sub-árvore direita e antes do rebalanceamento, no mínimo igual a
-2.

Quando o fator de balanceamento é igual a 2, existem duas
possibilidades (outras duas possibilidades, que existem quando o fator
de balanceamento é igual a -2, são análogas). 

\newcommand{\altura}{{\it altura\/}}

A primeira, mostrada no caso \verb+sobE+ abaixo, temos 
  $\text{\altura\ (\verb+ee+) = \altura\ (\verb+d+)}$.
Note que: 
  \begin{enumerate}
    \item se
      $\altura\ (\verb+ee+) < \altura\ (\verb+d+)$ então a
      árvore continuaria sendo uma árvore AVL após a inserção de um
      nodo em \verb+ee+;
    \item se $\altura\ (\verb+ee+) > \altura\ (\verb+d+)$ então a
      árvore já não seria uma árvore AVL antes da inserção de um nodo
      em \verb+ee+.
  \end{enumerate}
   
\begin{verbatim}
       Caso sobE      Caso sobED
           v              v
         /   \          /   \   
        ve    d        ve    d
       /  \           /  \                    
      ee  ed         ee   ved
                         / \
                       ede  edd                   

Árvore depois de rotacionada:

    Caso sobE          Caso sobED
     ve                   ved
    /  \                /     \ 
   ee   v             ve       v
       / \           /  \     / \ 
      ed  d         ee  ede edd  d

\end{verbatim}

O caso \verb+sobED+ pode ser expresso como \verb+sobE+ (aplicado ao
nodo com raiz \verb+ved+) seguido de \verb+sobD+ (aplicado ao mesmo
nodo).

%É importante notar que apenas um fator de balanceamento não é
%suficiente para determinar se uma árvore AVL necessita de rotação após
%uma inserção. Por exemplo, considere as duas árvores AVL a seguir:
%
%\begin{verbatim}
%      7                  7
%     /  \               /
%    3    8             3 
%   / \    \ 
%  2   5    9
% /     \ 
%1       6
%\end{verbatim}
%
%O fator de balanceamento da árvore com raiz \verb+7+ é igual a 1, nas
%duas árvores acima, antes da inserção.  No entanto, a inserção de
%\verb+4+ não quebra a condição de a árvore à esquerda ser AVL, ao
%contrário do que ocorre no caso da árvore à direita; após a inserção,
%e antes da rotação, que deve ser feita apenas na árvore à direita,
%temos:
%
%\begin{verbatim}
%     AVL              Não AVL 
%      7                  7
%     / \                /
%    3   8              3 
%   / \   \             \ 
%  2   5   9             4
% /   / \ 
%1   4   6
%\end{verbatim}

A tarefa de determinar, usando apenas o próprio fator de
balanceamento, a variação do fator de balanceamento após uma inserção,
e demonstrar que tal variação é verificada em todos os casos, é
deixada para trabalho futuro. Não encontramos na literatura textos que
abordam o assunto de forma clara e precisa.

O armazenamento da altura em cada nodo evita ter que calcular a altura
de cada nodo que está no caminho da árvore até o nodo inserido, o que
seria desnecessariamente ineficiente. 

\subsubsection{Versão funcional}

A inserção de elemento em árvore AVL é feita em Haskell como a seguir:

\begin{center}
\begin{tabular}{l}
\begin{hask}{ins}{\decremento}
module AVL (ArvoreAVL, arvVazia, ins) where

type Altura      = Integer
data ArvoreAVL a = Vazia | Nodo a Altura (ArvoreAVL a) (ArvoreAVL a)

arvVazia = Vazia

ins:: (Show a, Ord a) => a -> ArvoreAVL a -> ArvoreAVL a
ins k Vazia              = Nodo k 1 Vazia Vazia
ins k arv@(Nodo v _ e d) = 
 case compare k v of 
  LT -> let e1@(Nodo v1 a1 _ _) = ins k e 
            ad = altura d
         in if a1 - ad == 2 -- condição AVL precisa ser restaurada
            then if k < v1  
                 then sobE  (Nodo v undefined e1 d)  
                 else sobED (Nodo v undefined e1 d)
            else Nodo v (max a1 ad + 1) e1 d
  GT -> let d1@(Nodo v1 a1 _ _) = ins k d 
            ae = altura e
         in if a1 - ae == 2 -- condição AVL precisa ser restaurada 
            then 
               if k > v1
               then sobD  (Nodo v undefined e d1) 
               else sobDE (Nodo v undefined e d1) 
            else Nodo v (max ae a1 + 1) e d1
  _ -> arv

sobE :: ArvoreAVL a -> ArvoreAVL a
sobE (Nodo v _ (Nodo ve _ ee ed) d) = Nodo ve a ee (Nodo v ad ed d) 
  where ad = max (altura ed) (altura d) + 1
        a  = max (altura ee) ad         + 1

sobD :: ArvoreAVL a -> ArvoreAVL a
sobD (Nodo v _ e (Nodo vd _ de dd)) = Nodo vd a (Nodo v ae e de) dd
  where ae = max (altura e) (altura de) + 1
        a  = max ae         (altura dd) + 1

sobED :: ArvoreAVL a -> ArvoreAVL a
sobED (Nodo v _ (Nodo ve _ ee (Nodo ved _ ede edd)) d) = 
       Nodo ved a (Nodo ve ae ee ede) (Nodo v ad edd d)
  where a  = max ae           ad           + 1
        ae = max (altura ee ) (altura ede) + 1
        ad = max (altura edd) (altura d  ) + 1

sobDE :: ArvoreAVL a -> ArvoreAVL a
sobDE (Nodo v _ e (Nodo vd _ (Nodo vde _ dee ded) dd)) = 
       Nodo vde a (Nodo v ae e dee) (Nodo vd ad ded dd)
  where a  = max ae           ad           + 1
        ad = max (altura ded) (altura dd ) + 1
        ae = max (altura e  ) (altura dee) + 1 

altura (Nodo _ a _ _) = a
altura Vazia          = 0
\end{hask}
\end{tabular}
\end{center}

No pior caso, temos $T(h) = T(h-1) + k$, onde $h$ é a altura da árvore
e $k$ é o tempo de execução referente aos cálculos de i) altura de uma
sub-árvore, ii) condição AVL e iii) rotação (uma das funções
\inh{sobE}, \inh{sobED}, \inh{sobD}, \inh{sobDE}). Todos os tempos de
i) a iii) têm complexidade $O(1)$. Logo (cf.~seção
\ref{sec:maior-elemento}): $T(h) \asymp h$, ou seja, considerando que
$h \asymp lg n$ em uma árvore balanceada (onde $n$ é o número de
elementos da árvore):

  \[ T(n) \asymp lg n \]

\subsubsection{Versão imperativa}

\begin{center}
\begin{tabular}{l}
\begin{alg}{ArvoreBinariaDeInteiros}{White}
struct AVL
    int chave
    struct AVL *esq
    struct AVL *dir
    int alturaEsq, alturaDir
 
int altura (struct node *pnodo)
    if (pnodo == NULL)
        return 0
    return pnodo->altura
 
int max(int a, int b)
    if a > b return a else return b
 
struct node* novoNodo (int chave)
    struct AVL* nodo = (struct AVL*) malloc(sizeof(struct AVL))
    nodo->chave  = chave
    nodo->esq    = NULL;
    nodo->dir    = NULL;
    nodo->altura = 1 // novo nodo é folha
    return nodo
 
struct node *rightRotate(struct node *y)
    struct node *x = y->left;
    struct node *T2 = x->right;
 
    // Perform rotation
    x->right = y;
    y->left = T2;
 
    // Update heights
    y->height = max(height(y->left), height(y->right))+1;
    x->height = max(height(x->left), height(x->right))+1;
 
    // Return new root
    return x;
 
struct node *leftRotate(struct node *x)
    struct node *y = x->right;
    struct node *T2 = y->left;
 
    // Perform rotation
    y->left = x;
    x->right = T2;
 
    //  Update heights
    x->height = max(height(x->left), height(x->right))+1;
    y->height = max(height(y->left), height(y->right))+1;
 
    // Return new root
    return y;
 
int getBalance(struct node *N)
    if (N == NULL)
        return 0;
    return height(N->left) - height(N->right);
 
struct node* insert(struct node* node, int key)
    /* 1.  Perform the normal BST rotation */
    if (node == NULL)
        return(newNode(key));
 
    if (key < node->key)
        node->left  = insert(node->left, key);
    else
        node->right = insert(node->right, key);
 
    /* 2. Update height of this ancestor node */
    node->height = max(height(node->left), height(node->right)) + 1;
 
    /* 3. Get the balance factor of this ancestor node to check whether
       this node became unbalanced */
    int balance = getBalance(node);
 
    // If this node becomes unbalanced, then there are 4 cases
 
    // Left Left Case
    if (balance > 1 && key < node->left->key)
        return rightRotate(node);
 
    // Right Right Case
    if (balance < -1 && key > node->right->key)
        return leftRotate(node);
 
    // Left Right Case
    if (balance > 1 && key > node->left->key)
    {
        node->left =  leftRotate(node->left);
        return rightRotate(node);
    }
 
    // Right Left Case
    if (balance < -1 && key < node->right->key)
    {
        node->right = rightRotate(node->right);
        return leftRotate(node);
    }
 
    /* return the (unchanged) node pointer */
    return node;
\end{alg}
\end{tabular}
\end{center}

\subsection{Remoção}

\subsubsection{Versão funcional}

\subsubsection{Versão imperativa}

