%!TEX encoding = ISO-8859-1
\documentclass[a4paper,10pt]{book}

\usepackage{fancyheadings}
\usepackage{amsmath}
\usepackage{amssymb}
\usepackage{amsthm}
\usepackage{graphicx}
%\usepackage{fancybox}
\usepackage{makeidx}
\usepackage{subfigure}
\usepackage{indentfirst}
\usepackage{xcolor}
\usepackage[latin1]{inputenc}
\usepackage[portuguese]{babel}
\usepackage{comment}

%\usepackage{hevea}
%\usepackage{hevea-epsfbox-def}  

\usepackage{listings}
\lstloadlanguages{Haskell,[GNU]C++}

\definecolor{codecolor}{rgb}{0.74, 0.83, 0.9}
\definecolor{lightcyan}{rgb}{0.88, 1.0, 1.0}
\definecolor{lightblue}{rgb}{0.68, 0.85, 0.9}
\definecolor{myellow}{rgb}{0.99, 0.97, 0.37}
\lstdefinestyle{gstyle}
	{tabsize=2,
	 backgroundcolor=\color{codecolor},
	 frame=single,framerule=0pt,
	 emphstyle=\color{blue}\normalfont\itshape,
	 commentstyle=\color{red}\normalfont,
	 columns=flexible,
	 basicstyle=\color{black}\ttfamily,
	 identifierstyle=\normalfont\itshape,
	 xleftmargin=20pt,xrightmargin=0pt,
	 keepspaces=false
}

\lstset{style=gstyle}

\lstdefinelanguage[my]{Haskell}{morekeywords={data}}
\lstnewenvironment{hask}[1]{\lstset{language=[my]Haskell}}{}
\lstnewenvironment{alg}[2]{\lstset{language=[GNU]C++}}{}
\lstMakeShortInline[language={[my]Haskell}]�
\lstMakeShortInline[language={[GNU]C++}]�

\newcommand{\inCode}[1]
	{\begin{center}
      \colorbox{codecolor}{{#1}}
	\end{center}}

%\usepackage[T1]{fontenc}
%\selectlanguage{portuguese}


%\htmlhead{\sffamily\bfseries{Introdu��o a Algoritmos}}

\newcommand{\eqdef}{\overset{\text{def}}{=}}

%\newcommand{\graybox}[1]{\begin{minipage}{\textwidth}\psboxit{box 0.95 setgray fill}{
 % {\spbox{#1}}}\end{minipage}}

\makeindex

\setcounter{tocdepth}{3}

%----- text format parameters -----
\setlength\textwidth{150mm}
\setlength\textheight{241mm}
\setlength\oddsidemargin{5.4mm}
\setlength\evensidemargin{5.4mm}
%\setlength\marginparsep{5mm}
%\setlength\marginparwidth{15mm}
\addtolength\voffset{-5mm}
\addtolength\topmargin{-3mm}
\addtolength\headsep{2mm}
\setlength\footskip{10mm}

%---------------- pagestyle ----------
\pagestyle{fancyplain}
\addtolength\headwidth{\marginparsep}
\renewcommand{\chaptermark}[1]
             {\markboth{#1}{}}
\renewcommand{\sectionmark}[1]
             {\markright{\thesection\ #1}}
\lhead[\fancyplain{}{\bfseries\thepage}]
      {\fancyplain{}{\bfseries\rightmark}}
\rhead[\fancyplain{}{\bfseries\leftmark}]
      {\fancyplain{}{\bfseries\thepage}}

\cfoot{}

%----- lista de exercicios ---------------
%\newenvironment{exerc}
%   {\renewcommand{\labelenumi}{\textbf{\theenumi.}}
%    \begin{enumerate}
%   }
%   {\end{enumerate}}                    
\renewcommand{\chaptername}{Cap\'{\i}tulo}
\renewcommand{\tablename}{Tabela}
\renewcommand{\figurename}{Figura}
\renewcommand{\contentsname}{Sum\'ario}

\begin{document}

\pagestyle{empty}
\begin{center}
\begin{tabular}{c}
\Huge{\sffamily\bfseries{Introdu��o a Algoritmos}}\\
\end{tabular}
\end{center}

\begin{tabular}{l}
{\Large{\sffamily\bfseries{Carlos Camar�o}}} \\
{\small Universidade Federal de Minas Gerais}\\
{\small Doutor em Ci\^encia da Computa\c{c}\~ao pela Universidade de Manchester, Inglaterra} \\ \\
\end{tabular}

\begin{tabular}{l}
Direitos exclusivos\\
Copyright \copyright\ 2009 by Carlos Camar\~ao
\end{tabular}

� permitida a duplica��o ou reprodu��o, no todo ou em parte, sob
quaisquer formas ou por quaisquer meios (eletr�nico, mec�nico,
grava��o, fotoc�pia, distribui��o na Web ou outros), desde que seja
para fins n�o comerciais.

\pagestyle{fancy}
\pagenumbering{roman}
%\setcounter{page}{5}

%\renewcommand{\chaptername}{Cap�tulo}
%\renewcommand{\tablename}{Tabela}
%\renewcommand{\figurename}{Figura}

%\renewcommand{\contentsname}{Sum�rio}

%\tableofcontents

\setcounter{secnumdepth}{-2}
%\chapter{Pref�cio}

Este livro prov� uma introdu��o ao estudo de algoritmos. Ele apresenta
estruturas de dados b�sicas e algoritmos de pesquisa e ordena��o, e
prov� uma introdu��o ao importante ramo da ci�ncia da computa��o que
trata do desenvolvimento e an�lise da efici�ncia de algoritmos. O
assunto da an�lise de efici�ncia � comumente chamado em computa��o de
complexidade.

Essa an�lise aborda em geral quanto tempo � gasto na execu��o de um
algoritmo em fun��o do tamanho da entrada: diz-se complexidade de
tempo do algoritmo. Al�m do tempo, pode ser analisada tamb�m a
complexidade de espa�o (quanto espa�o de mem�ria � gasto na execu��o
em fun��o do tamanho da entrada).

... nota��o ...

... funcional ...

... clareza, concis�o ...

\subsection{Conte�do e Organiza��o do Livro}

\subsection{Recursos Adicionais}

\subsection{Pr�-requisitos}

Os pr�-requsitos s�o:

\begin{enumerate}

\item Experi�ncia inicial com provas por indu��o. (??)

\end{enumerate}


\input{meta.keys}

\pagestyle{fancy}
\setcounter{secnumdepth}{10}
\pagenumbering{arabic}

\chapter{Introdu��o}
\label{Introducao}

Este livro prov� uma introdu��o ao estudo de algoritmos. Ele apresenta
uma introdu��o a estruturas de dados b�sicas e algoritmos de pesquisa
e ordena��o.

Vamos identificar neste livro algoritmo com fun��o, no sentido de
prover uma sequ�ncia de passos que associa a cada valor de (um
conjunto de valores de) entrada um �nico valor de (um conjunto de
valores de) sa�da.

A diferen�a que existe entre o conceito usual de fun��o � a nota��o
usualmente empregada para especifica��o da sequ�ncia de passos. Em
computa��o, � usual o emprego de uma nota��o ou linguagem {\em
  imperativa\/}, ao passo que usualmente defini��es de fun��es
empregam uma nota��o mais {\em declarativa\/}, ou {\em funcional\/}.

\section{Ordena��o}

\index{ordena\c{c}\~ao}}
Por exemplo, considere o problema de ordena��o, especificado
formalmente como a seguir (um problema computacional especifica a
rela��o que deve existir entre a entrada e a sa�da):

\Entrada: sequ�ncia de elementos $S_0$.

\Saida: sequ�ncia de elementos ordenada $S$ tal que $S$ � uma
permuta��o de $S_0$.

Uma sequ�ncia $a_1, \ldots, a_n$ � ordenada se $a_i \leq a_{i+1}$ para
$i=1,\ldots, n-1$.

\subsection{Sobre Permuta��o} 

\index{permuta\c{c}\~ao}
Uma permuta��o (ou arranjo) � uma redisposi��o de (um conjunto ou
sequ�ncia de) elementos em uma certa sequ�ncia (se contrap�e a uma
{\em combina��o}, na qual a ordem dos elementos resultantes n�o �
relevante). Por exemplo, h� 6 permuta��es distintas dos elementos
1,2,3, que s�o, escritas como tuplas: (1,2,3), (1,3,2), (2,1,3),
(2,3,1), (3,1,2), (3,2,1).

Outro nome, usado no contexto de palavras, � {\em anagrama}. 

\index{anagrama}
Um anagrama � o resultado de rearranjar as letras de uma palavra ou
frase para produzir uma nova palavra ou frase, usando todas as letras
originais exatamente uma vez. Por exemplo, "ovo" pode ser rearranjado
para "voo".

Permuta��es ocorrem em diversas �reas da matem�tica e proeminentemente
no estudo de algoritmos, particularmente de ordena��o, em ci�ncia da
computa��o.

O n�mero de permuta��es de $n$ elementos distintos � igual ao fatorial
de $n$ (usualmente escrito em matem�tica como $n!$), que � igual ao
produto de todos os inteiros positivos menores ou iguais a $n$.

\section{Ordena��o por Inser��o}

\index{ordena\c{c}\~ao!por inser\c{c}\~ao}
Um algoritmo ou fun��o que resolve o problema de ordena��o
especificado acima, chamado de {\em ordena��o por inser��o\/}, �
mostrado a seguir. Ele reflete o modo como um jogador de baralho
usualmente ordena uma sequ�ncia de cartas recebidas (por exemplo, em
um jogo de buraco).

\index{Haskell}
Neste livro, usamos sempre pseudo-c�digos com nota��o funcional e
imperativa para representar cada algoritmo. A nota��o funcional � a
linguagem \Haskell\ e a nota��o imperativa � um pseudo-c�digo
semelhante a \C, \Pascal\ ou \Java.

O uso da nota��o funcional pode ser desconsiderado em cursos que
desejam abordar apenas o paradigma de programa��o imperativo.

\subsection{Vers�o funcional}
\label{insertion-sort-func}

A nota��o funcional ser� explicada sempre que necess�rio, isto �,
sempre que houver alguma possibilidade de d�vida. Uma descri��o
sucinta da linguagem Haskell � inclu�da no Ap�ndice \ref{Ap-Haskell}
(o leitor n�o familiarizado com Haskell deve ler o Ap�ndice
\ref{Ap-Haskell}).  Descri��es mais completas de Haskell podem ser
encontradas, por exemplo, em
\cite{PeytonJones92,Thompson99,O'Sullivan:2008:RWH,Lipovaca:2011:LYH}.

\newcommand{\elem}{{\it elem\/}}
\newcommand{\insert}{{\it insert\/}}
\newcommand{\sort}{{\it sort\/}}
\newcommand{\delete}{{\it delete\/}}
\newcommand{\perm}{{\it is\_a\_permutation\_of\/}}
\newcommand{\sorted}{{\it sorted\/}}
\newcommand{\Pair}{{\it Pair\/}}

\index{\sort}
\index{\insert}
\progb{
      \sort\ []       \hspace*{1cm} = []\\
      \sort\ ($a$:$x$)\             = \insert\ $a$ (\sort\ $x$)\\
      \hspace*{1cm}\\
      \insert\ $a$ []       \hspace*{1cm} = []\\
      \insert\ $a$ ($b$:$x$) \\ 
          \hspace*{.2cm} | $a$ <= $b$  \hspace*{1.2cm} = $a$: ($b$ : $x$)\\
          \hspace*{.2cm} | \otherwise  \hspace*{.7cm} = $b$: \insert\ $a$ $x$
}

Explica��es sobre a nota��o funcional (usada em Haskell):

\begin{enumerate}

\item $f$ $x$ (aplica��o funcional --- a base da programa��o
  funcional) � o mesmo que {\tt $f$($x$)} (mas melhor porque evita os
  par�nteses).

\item \index{aplica\c{c}\~ao funcional} {\tt $b$: \insert\ $a$ $x$} �
  o mesmo que {\tt $b$: (\insert\ $a$ $x$}): a aplica��o funcional tem
  preced�ncia sobre o uso de operadores bin�rios.

\item \index{operador!bin\'ario} O uso de um operador bin�rio nada
  mais � do que uma varia��o sint�tica de (a��car sint�tico para) uma
  aplica��o funcional; o uso de um operador bin�rio pode ser
  transformado em uma aplica��o funcional, e vice-versa. Para
  transformar um operador bin�rio em uma aplica��o funcional, basta
  colocar o operador entre par�nteses, e para transformar uma
  aplica��o funcional em um operador, basta colocar o nome da fun��o
  entre crases.  Exemplos:

  \begin{tabular}{lll} 
    {\tt 2 + 3} & � equivalente a & {\tt (+) 2 3} \\ 
    {\tt $b$ : $x$} & � equivalente a & {\tt (:) $b$ $x$} \\ 
    $f$ $x$ $y$ & � equivalente a & {\tt $x$ `$f$` $y$} 
  \end{tabular}

\item \index{Tipo!recursivo} \index{\List}\index{Cons}\index{Nil} As
  fun��es \insert\ and \sort\ usam listas, um tipo recursivo, que � um
  tipo de dado alg�brico (chamado em Haskell de {\tt data}) parecido
  com o seguinte:

    \progb{\data\ \List\ $a$ = \Nil\ | \Cons\ $a$ (\List\ $a$)}

  \index{Tipo de dado!alg�brico} Um tipo de dado alg�brico � a maneira
  como se definem somas (de tipos, sendo que s� podem existir somas
  disjuntas de tipos), que modelam escolha (``ou'') de tipos de dados.

  \index{polimorfismo}\index{Tipo!polim�rfico}
  A declara��o de \List\ acima especifica que um valor de tipo lista �
  polim�rfico (o uso da vari�vel de tipo $a$ indica que \List\ � um
  construtor de tipos que pode ser aplicado a {\em qualquer\/} tipo
  $t$, isto �, podemos ter qualquer inst�ncia \List\ $t$, para {\em
    qualquer\/} tipo $t$), e que uma lista (um valor de tipo
  $\List\ t$, para algum tipo $t$) pode ser \Nil\ (uma lista vazia)
  {\em ou\/} {\tt \Cons\ $v$ $x$}, uma lista (n�o vazia) formada por
  um valor $v$ (cabe�a da lista) e de um restante (ou rabo) da lista,
  $x$ (que deve ser do mesmo tipo da lista da qual � o restante).

  \index{Listas}\index{Cons}\index{Nil}
  O tipo de listas em Haskell (� parecido mas) difere ligeiramente do
  tipo alg�brico acima porque o construtor \Nil\ � escrito como {\tt
    []} e o construtor \Cons\ � escrito como um operador bin�rio {\tt
    :}. Assim, em vez de escrever, {\tt \Cons\ 1 \Nil}, escreve-se em
  Haskell {\tt 1:[]}. Al�m disso, pode-se escrever tamb�m {\tt
    [1,2,3]} em vez de {\tt 1:2:3:[]} --- i.e.~em vez de {\tt
    1:(2:(3:[]))}.

\item \index{soma!disjunta} \index{currifica��o} Tipos de dados
  alg�bricos permitem definir somas (disjuntas) de tipos, que modelam
  escolha (``ou'') de tipos de dados. Para definir produtos de tipos,
  podemos usar tipos alg�bricos, que permitem produtos
  ``linearizados'' (tamb�m chamados de ``currificados'') ou produtos
  cartesianos (generalizados), tamb�m chamados de tuplas.

  \index{\Pair} Por exemplo:

  \progb{\data\ \Pair\ $a$ $b$ = \Pair\ $a$ $b$}

  define um construtor de tipos \Pair, que tem dois par�metros que
  podem ser instanciados para quaisquer tipos $t$ e $t'$: por exemplo,
  \Pair\ \Int\ \Bool\ representa pares de valores de inteiros e
  booleanos (o primeiro componente do par � um inteiro e o segundo um
  valor booleano). � semelhante ao produto {\tt (\Int,\Bool)}. A
  diferen�a � que valores do primeiro s�o constru�dos da forma {\tt
    \Pair\ 1 \True} (especificando um valor inteiro e em seguida um
  valor booleano), ao passo que valores do segundo s�o constru�dos da
  forma {\tt (1,\True)} (especificando, entre par�nteses, primeiro um
  valor inteiro, seguido de uma v�rgula, e depois um valor booleano).

\end{enumerate}

Para mostrar a corre��o de \sort, podemos usar predicados (fun��es de
contra-dom�nio \Bool); vamos provar:

  \[ \text{\it \sort\ $x$ = $y$ implica:} \]

  \begin{enumerate}

    \item \sorted\ $y$
    \item $x$ `\perm` $y$

 \end{enumerate}

Temos:

\index{\sorted}
\index{\perm}
  \progb{
        \sorted\ []            \hspace*{2cm}= \True\\
        \sorted\ [$a$]         \hspace*{1.7cm}= \True\\
        \sorted\ ($a$:$b$:$x$) \hspace*{.1cm}= ($a$ <= $b$) \&\& \sorted\ ($b$:$x$) \\ 
        \hspace*{1cm} \\  
        [] \symbol{96}\perm\symbol{96} []         \hspace*{1.3cm} = \True\\
        ($a$:$x$) \symbol{96}\perm\symbol{96} $y$ \hspace*{.1cm} = \elem\ $a$ $y$ \&\& 
                                ($x$ \symbol{96}\perm\symbol{96} (\delete\ $a$ $y$)) \\
        \hspace*{1cm} \\  
        $a$ \symbol{96}\elem\symbol{96}\ []        \hspace*{1cm} = \False\\
        $a$ \symbol{96}\elem\symbol{96}\ ($b$:$x$) \hspace*{.1cm}= ($a$ == $b$) || ($a$ \symbol{96}\elem\symbol{96} $x$)\\
        \hspace*{1cm} \\  
        \delete\ $a$ []         \hspace*{0.7cm} = []\\
        \delete\ $a$ ($b$:$x$)\\
           \hspace*{.2cm}| $a$==$b$     \hspace*{2cm}  = $x$ \\
           \hspace*{.2cm}| \otherwise\  \hspace*{0.7cm} = $b$: \delete\ $a$ $x$
  }

Explica��es sobre a nota��o funcional (usada em Haskell):

\begin{enumerate}

\item \index{{\tt \&\&}\index{{\tt
      ||}}\index{conjun��o}\index{disjun��o} {\tt \&\&} e {\tt ||} s�o
  operadores bin�rios l�gicos de conjun��o e disjun��o,
  respectivamente.

\item \index{guarda} A defini��o de \delete\ usa {\em guardas\/}, que
  s�o express�es booleanas usadas na defini��o de fun��es; para cada
  chamada de fun��o, a primeira (na ordem textual) guarda cuja
  avalia��o retorna \True\ define o resultado da chamada da fun��o,
  pela avalia��o da express�o associada a essa guarda (que segue o
  s�mbolo {\tt =}). Por exemplo, a guarda na defini��o de \delete\ �
  equivalente a: \iif\ $a$==$b$ \tthen\ $x$ \eelse\ $b$: \delete\ $a$
  $x$.

\end{enumerate}

Prova: O caso base sai diretamente e o caso indutivo � consequ�ncia
dos seguintes lemas:

Lema 1: Para todo $a,x$, 
        {\tt \sorted ($x$)} implica {\tt \sorted(\insert\ $a$ $x$)}

Lema 2: Para todo $a,x$, 
        {\tt \sort\ $x$ \symbol{96}\perm\symbol{96} $x$} implica 
        {\tt \insert\ $a$ (\sort\ $x$) \symbol{96}\perm\symbol{96} ($a$:$x$)}

% ... prova dos lemas?

\subsection{Vers�o imperativa}
\label{insertion-sort-imperativ}

A vers�o imperativa usa o pr�prio arranjo para a ordena��o (nenhum
outro arranjo ou estrutura de dados auxiliar) e a seguinte ideia:

  \begin{quotation}
     insere {\tt $A$[$j$]} no arranjo ordenado de {\tt $A$[1]} at�
     {\tt $A$[$j$-1]}, de $j=2$ at� o tamanho do arranjo
  \end{quotation}

A ideia d� origem ao seguinte algoritmo, escrito em pseudo-c�digo como
(note que endenta��o no pseudo-c�digo indica aninhamento na estrutura
de blocos):

\newcommand{\key}{{\it key\/}}

\index{\sort}
\progb{
\sort($A$) \{ \\
\hspace*{.5cm} \for\ j $\leftarrow$ 2 \tto\ \length[$A$] \do\\
\hspace*{1cm}     \key\ $\leftarrow$ $A$[j]\\
\hspace*{1cm}     /* Insere $A$[$j$] no arranjo ordenado $A$[1..$j$-1] */ \\
\hspace*{1cm}     $i$ $\leftarrow$ $j$-1\\
\hspace*{1cm}     \while\ ($i$ > 0 \&\& $A$[$i$]>\key) \do\\
\hspace*{2cm}        $A$[$i$+1] $\leftarrow$ $A$[$i$]\\
\hspace*{2cm}        $i$ $\leftarrow$ $i$ - 1\\
\hspace*{1cm}    $A$[$i$+1] $\leftarrow$ \key\\
\}
}

\index{invariante}
A corre��o do algoritmo adv�m de que o invariante, especificado
bastante informalmente como:

  \begin{quotation}
    --- no in�cio da execu��o de cada itera��o do comando \for, o
    sub-arranjo {\tt $A$[1..$j$-1]} cont�m os elementos que estavam
    originalmente nesse sub-arranjo, mas de forma ordenada ---
  \end{quotation}

� verdadeiro no in�cio (antes da execu��o da primeira itera��o do
\for), antes e ap�s cada itera��o, e no final, quando ent�o a
termina��o garante a corre��o do algoritmo (ordenamento de todo o
arranjo).

� importante observar que a transforma��o dessa prova informal em uma
prova formal � relativamente muito mais dif�cil do que no caso
funcional.

No pr�ximo cap�tulo vamos introduzir introduzir a nota��o e os
conceitos principais usados para an�lise da complexidade (efici�ncia)
de algoritmos, para que possamos analisar a complexidade de algoritmos
(come�ando pela complexidade dos algoritmos apresentados neste
cap�tulo).



%%!TEX encoding = ISO-8859-1
\chapter{Complexidade}
\label{Complexidade}

Programas de computadores est�o sendo usados em diversas aplica��es,
que est�o expandindo sua atua��o cada vez mais e para �reas cada vez
mais diversas. Al�m disso, a ci�ncia da computa��o vem influenciando
as mais diversas �reas cient�ficas, como a l�gica, a matem�tica, a
f�sica etc.

A ci�ncia da computa��o se preocupa n�o s� com a corre��o de
algoritmos --- o fato de que um algoritmo computa o resultado
especificado ou esperado para cada inst�ncia dos dados de entrada ---
mas tamb�m com a sua efici�ncia, ou {\em complexidade}. Efici�ncia se
mede usualmente pelo tempo gasto na execu��o de um programa, mas pode
alternativamente ser baseada em uma medida do espa�o (quantidade de
mem�ria).

A efici�ncia ou complexidade � em geral medida como uma fun��o de uma
medida que especifica o tamanho da entrada. Em geral um algoritmo
eficiente � um algoritmo cujo tempo e espa�o gastos na execu��o n�o
aumentam muito com o aumento do tamanho da entrada.

Em geral, problemas dif�ceis de serem resolvidos eficientemente s�o
aqueles para os quais os algoritmos conhecidos e usados gastam um
tempo de execu��o cuja varia��o com o tamanho da entrada n�o �
polinomial (n�o tem como limite superior um polin�mio, $n^k$, onde $n$
� o tamanho da entrada e $k$ � um valor constante, que independe de
$n$), e sim exponencial (ou seja, a varia��o do tempo de execu��o com
o tamanho da entrada � uma fun��o do tipo $k^n$, onde $k$ � uma
constante, como veremos em geral $k=2$). Algoritmos de complexidade
exponencial (isto �, cujo tempo de execu��o aumenta exponencialmnte
com o aumento do tamanho da entrada) s�o ditos {\em intrat�veis\/}.

\section{Complexidade de Fun\c{c}\~oes}
\label{complexidade-de-funcoes}

O tempo de execu��o de um algoritmo sequencial � a soma dos tempos de
execu��o de cada passo executado (no caso de um algoritmo imperativo,
cada comando executado). Cada passo pode ser executado um certo n�mero
$k$ de vezes, e essa soma considera ent�o, para cada passo, o produto
$k\times c$, onde $c$ � uma medida do tempo gasto para execu��o do
passo uma �nica vez.

Na maioria das vezes, estamos interessados em determinar o tempo de
execu��o do algoritmo para o {\em pior caso\/} de uma entrada de certo
tamanho. Isso � devido ao fato de que i) o pior caso � um limite
superior (que nunca poder� ser ultrapassado), ii) para muitos
algoritmos o pior caso ocorre bastante frequentemente e iii)
frequentemente o pior caso representa valor pr�ximo do caso m�dio, e
iv) em geral, o {\em caso m�dio\/} � mais dif�cil de ser analisado,
envolvendo t�cnicas de an�lise de probabilidades.

A an�lise de complexidade envolve em geral considerar apenas o termo
mais significativo de uma soma de termos que expressa uma fun��o sobre
o tamanho da entrada (em geral usa-se os n�meros naturais como medida
desse tamanho). Por exemplo, se a fun��o que expressa a varia��o do
tempo de execu��o com o tamanho da entrada � expressa como $f(n) =
an^2 + bn + c$, onde $a,b,c$ s�o constantes, dizemos que a fun��o tem
ordem de complexidade $n^2$, que � o termo mais significativo do
polin�mio $f$. Os termos de menor ordem no polin�mio s�o relativamente
pouco significantes para valores de $n$ grandes, e tamb�m ignora-se as
constantes (ignoram-se todos os termos de menor ordem e ignora-se
tamb�m a constante $a$ do termo de maior ordem), pelo mesmo motivo de
ser relativamente pouco significante para valores grandes. Dizemos
ent�o que $f$ � uma fun��o de complexidade quadr�tica (o termo de
maior ordem no polin�mio que define a complexidade � $n^2$).

\section{Efici�ncia assint�tica}
\label{eficiencia-assintotica}

Ao considerar a varia��o da efici�ncia de acordo com tamanhos de
entrada grandes, fazemos simplifica��es para estudar a efici�ncia ou
complexidade de algoritmos, e chamamos a complexidade de {\em
  assint�tica\/}. Usualmente, considera-se algoritmos mais eficientes
que outros considerando a complexidade assint�tica.

A nota��o geralmente usada na descri��o da complexidade assint�tica de
algoritmos e programas � apresentada a seguir.

\subsection{Nota��o $\Theta$}
\label{Notacao-Theta}

Considera-se o tamanho da entrada como um valor natural (i.e.~o
dom�nio de fun��es de complexidade � igual ao conjunto dos naturais,
$\mathbb{N}$) e o contra-dom�nio � o conjunto dos n�meros reais
positivos ($\mathbb{R}^+$). 

Seguimos a terminologia usual, que usa $f(n)$ (digamos, por exemplo,
$lg n$) para se referir na verdade � fun��o $f$ (em nota��o de
$\lambda$-calculus, � fun��o $\lambda n \rightarrow lg n$). Temos:

\[ \begin{array}{ll}
     \Theta(g(n)) = \{ f(n) : & \text{ existem } c_1, c_2, n_0 \text{ positivos tais que } \\
                              & c_1g(n) \leq f(n) \leq c_2g(n) \text{ para todo } n\geq n_0 
   \end{array}
\]

Em palavras, $f(n)$ � membro de $\Theta(g(n))$ se $f(n)$ sempre est�
entre $c_1g(n)$ e $c_2g(n)$, para alguns $c_1, c_2$, para $n$
suficientemente grande (i.e.~a partir de algum $n_0$). 

\ldots Gr�fico? 

Como $\Theta(g(n))$ � um conjunto, o correto seria escrever $f(n) \in
\Theta(g(n))$, mas � usual escrever $f(n) = \Theta(g(n))$, um abuso de
nota��o, que simplifica a nota��o. 

Escrevemos tamb�m $f(n) \asymp g(n)$ (seguindo Minko Markov
\cite{Minko-Markov-2013}) como sin�nimo de $f(n) = \Theta(g(n))$.

%ssume-se tamb�m, na defini��o de $\Theta(g(n))$ que $g(n)$ e os
%embros $f(n)$ s�o assintoticamente n�o-negativos para $n$
%uficientemente grande.

Por exemplo, temos:

\begin{enumerate}

\item $an^2 - bn \asymp n^2$, para quaisquer determinadas constantes
      $a,b$.

      Para mostrar isso, devemos determinar $c_1,c_2,n_0$ tais que $0
      \leq c_1(n^2) \leq an^2 - bn \leq c_2(n^2)$, para $n\leq n_0$.
      Ou seja (dividindo por $n^2$):

        \[ c_1 \leq a -\frac{b}{n} \leq c_2 \text{, para } n\leq n_0 \]
      Essa desigualdade pode ser satisfeita, para todo $n\geq 1$,
      tomando $c_1 \leq a - b$ e $c_2 \geq a$. 

\item $an \not\asymp n^2$, para qualquer constante positiva $a$.

      Nesse caso, dever�amos ter $c_1n^2 \leq an$, para $n$
      suficientemente grande, ou seja, $c_1n \leq a$, o que n�o
      acontece.

\item $an^3 \not\asymp n^2$, para qualquer constante positiva $a$.

      Nesse caso, dever�amos ter $c_2n^2 \geq an^3$, para $n$
      suficientemente grande, ou seja, $c_2 \leq an$, o que n�o
      acontece.

\end{enumerate}

$\Theta(1)$ � usualmente usado em vez de $\Theta(n^0)$,
considerando-se claro pelo contexto a vari�vel que est�-se
considerando como medida do tamanho da entrada.

\section{Nota��o $O$}
\label{Notacao-O}

A nota��o $\Theta$ limita assintoticamente uma fun��o por limites
superior e inferior. A nota��o $O$ usa estabelece apenas um limite
superior:

\[ \begin{array}{ll}
     O(g(n)) = \{ f(n) : & \text{ existem } c, n_0 \text{ positivos tais que } \\
                              & f(n) \leq cg(n) \text{ para } n\geq n_0 
   \end{array}
\]

Em palavras, $f(n)$ � membro de $O(g(n))$ se $f(n)$ sempre � menor ou
igual a $cg(n)$, para algum $c$, e para $n$ suficientemente grande
(i.e.~a partir de algum $n_0$).
 
Assim como na nota��o $\Theta$, usa-se $f(n) = O(g(n))$ em vez de
$f(n) \in O(g(n))$. 

Escrevemos tamb�m $f(n) \preceq g(n)$ como sin�nimo de $f(n) =
O(g(n))$ (seguindo Minko Markov \cite{Minko-Markov-2013}).

Note que $f(n) \asymp g(n)$ implica $f(n) \preceq g(n)$, mas a
implica��o inversa n�o � verdadeira.

Por exemplo, $an + b \preceq n^2$ mas $an + b \npreceq \Theta(n^2)$.

Usando a nota��o $O$ podemos frequentemente ter uma boa ideia do
limite superior para o tempo de execu��o de um programa pela inspec��o
de sua estrutura de repeti��o.  Por exemplo, o aninhamento duplo do
programa imperativo de ordena��o por inser��o indica um limite
superior $O(n^2)$ para o pior caso do tempo de execu��o, uma vez que o
custo de cada comando interno ao comando \while\ � $O(1)$, as
repeti��es \for\ e \while\ s�o controladas pelos �ndices $i$, $j$ que
variam no m�ximo at� $n$, e a repeti��o mais interna � executada no
m�ximo uma vez para cada par de $n^2$ valores $(i,j)$.

Note que a nota��o $O$ fornece um limite superior, e portanto � v�lida
para qualquer entrada. Isso n�o ocorre com a nota��o $\Theta$, uma vez
que podem existir entradas para as quais o algoritmo se comporta mais
eficientemente do que no caso do limite inferior. Por exemplo,
$O(n^2)$ � v�lido para qualquer entrada do algoritmo de ordena��o por
inser��o (� um limite superior), mas existem entradas
(especificamente, se a entrada j� est� ordenada) para as quais o tempo
de execu��o � linear, dado por $\Theta(n)$. 

Dizer que o tempo de execu��o do algoritmo de ordena��o por inser��o �
$O(n^2)$ significa, portanto, que existe uma fun��o $f$, em $O(n^2)$,
tal que, para qualquer entrada e qualquer $n$, o tempo de execu��o do
programa para essa entrada � limitado a $f(n)$ (n�o significa que o
varia��o do tempo de execu��o do algoritmo para uma entrada particular
varia quadraticamente com o tamanho da entrada, mas que pode variar no
m�ximo quadraticamente com o tamanho da entrada).

\section{Nota��o $\Omega$}
\label{Notacao-Omega}

A nota��o $\Omega$ limita assintoticamente uma fun��o por um limite
inferior:

\[ \begin{array}{ll}
     \Omega(g(n)) = \{ f(n) : & \text{ existem } c, n_0 \text{ positivos tais que } \\
                              & cg(n) \leq f(n) \text{ para } n\geq n_0 
   \end{array}
\]

Em palavras, $f(n)$ � membro de $\Omega(g(n))$ se $f(n)$ sempre �
maior ou igual a $cg(n)$, para algum $c$, e para todo $n$
suficientemente grande.

Assim como na nota��o $\Theta$, usa-se $f(n) = \Omega(g(n))$ em vez de
$f(n) \in \Omega(g(n))$.

Escrevemos tamb�m $f(n) \succeq g(n)$ como sin�nimo de $f(n) =
\Omega(g(n))$ (seguindo Minko Markov \cite{Minko-Markov-2013}).

Das defini��es, � f�cil ver que, para quaisquer fun��es $f,g$, temos
$f(n) \asymp g(n)$ se e somente se $f(n) \preceq g(n)$ e $f(n) \succeq
g(n)$.

A nota��o $\Omega$ estabelece um limite inferior, e envolve, assim,
an�lise do comportamento do algoritmo para o melhor caso dos dados de
entrada. N�o consideraremos an�lises de comportamento de algoritmos no
melhor caso neste livro, e a nota��o $\Omega$ ter� assim aplica��o
limitada, sendo usada mais como complementa��o � nota��o $O$.

\section{Uso de nota��o assint�tica em f�rmulas}
\label{uso-notacao-assintotica}

Em equa��es do tipo $n = O(n^2)$, a nota��o $O$ significa $n \in
O(n^2)$. 

Em geral, no entanto, a ocorr�ncia da nota��o assint�tica expressa uma
fun��o qualquer, an�nima, para a qual n�o h� interesse em especificar
um nome. Por exemplo, $an^2 + bn + c$ pode ser expressa como $an^2 +
\Theta(n)$, siginificando $an^2 + f(n)$, onde $f(n)$ � um membro de
$\Theta(n)$, no caso $bn + c$. Continuando nesse sentido, $an^2 +
\Theta(n)$ pode ser expressa como $\Theta(n^2)$. 

Essas abrevia��es s�o usadas para evitar ter que escrever fun��es que
correspondem a termos de menor grau em f�rmulas.

Como exemplo de uso de nota��o assint�tica, temos: $O(2^{O(lg
  n)})=n^{O(1)}$.  O uso da nota��o $O$ nessa f�rmula expressa
supress�o de constantes. Note que $n \preceq 2^{lg n}$, e portanto,
para qualquer constante $c$, temos: $n^c \asymp 2^{c lg n}$, e $O(2^{O(lg
  n)})=n^{O(1)}$ � outra forma de expressar essa igualdade.

\section{Nota��es $o$, $\omega$}
\label{Notacao-o}

Define-se $o(g(n))$ para indicar que trata-se de uma aproxima��o
assint�tica estrita:

\[ \begin{array}{ll}
     o(g(n)) = \{ f(n) : & \text{ para todo } c \text { positivo existe } n_0 \text{ positivo tal que } \\
                              & f(n) < cg(n) \text{ para } n\geq n_0 
   \end{array}
\]

Em palavras, $f(n)$ � membro de $O(g(n))$ se $f(n)$ sempre �
estritamente menor que $cg(n)$, para todo $c$, e para $n$
suficientemente grande (i.e.~a partir de algum $n_0$).
Isso � equivalente a:

  \[ \llim_{n\rightarrow\!\infty} \frac{f(n)}{g(n)} = 0 \]

Analogamente, a nota��o $\omega$ indica um limite inferior que �
assintoticamente estrito ($\omega$ est� para $\Omega$ assim como $o$
est� para $O$):

\[ \begin{array}{ll}
     \omega(g(n)) = \{ f(n) : & \text{ para todo } c \text { positivo existe } n_0 \text{ positivo tal que } \\
                              & cg(n) < f(n) \text{ para } n\geq n_0 
   \end{array}
 \]

Isso � equivalente a:

  \[ \llim_{n\rightarrow\!\infty} \frac{g(n)}{f(n)} = 0 \]

Escrevemos tamb�m $f(n) \succ g(n)$ como sin�nimo de $f(n) = o(g(n))$,
e $f(n) \prec g(n)$ como sin�nimo de $f(n) = \omega(g(n))$ (seguindo
Minko Markov \cite{Minko-Markov-2013}).

Por exemplo, para qualquer $a>0$, $an^2 \succ n$ mas $an^2 \nsucc
n^2$.

\section{Propriedades de Rela��es assint�ticas}
\label{Propriedades-de-relacoes-assintoticas}

Para todas as rela��es $R$ iguais a $\asymp$, $\preceq$, $\succeq$,
$\prec$, $\succ$, a seguinte transitividade ocorre:

  \[ f(n) R g(n), g(n) R h(n) \text { implicam } f(n) R h(n) \]

Ocorre tamb�m reflexividade para $R = \asymp$, $\preceq$, $\succeq$:

  \[ f(n) R f(n) \]

Para $\asymp$ ocorre simetria (mas n�o para as demais rela��es):

  \[ f(n) \asymp g(n) \text{ se e somente se } g(n) \asymp f(n) \]

Para $\succeq$ e $\preceq$, e $\succ$ e $\prec$, temos:

  \[ \begin{array}{l}
       f(n) \succeq g(n) \text{ se e somente se } g(n) \preceq f(n) \\
       f(n) \succ g(n) \text{ se e somente se } g(n) \prec f(n) 
     \end{array}
  \]

Essas propriedades s�o similares �s verificadas para as rela��es de
igualdade e desigualdade entre n�meros reais, motivam o uso das
nota��es semelhantes para fun��es e motivam chamar $f$ de
assintoticamente menor que $g$ se $f(n) = o(g(n))$, $f$ de
assintoticamente maior que $g$ se $g(n) = \omega(f(n))$), e
analogamente para ($\asymp$ e $\Theta$, assintoticamente igual),
($\preceq$ e $O$, assintoticamente menor ou igual a) e ($\succeq$ e
$\Omega$, assintoticamente maior ou igual a).

\section{Complexidade polinomial, exponencial e logar�tmica}
\label{Complexidade-polinomial-exponencial-logaritmica}

Um polin�mio de grau $k$, sendo $k$ inteiro n�o negativo, � uma fun��o
da forma:

  \[ \sum_{i=0}^{k} a_in^i \]
onde $a_i$ s�o constantes inteiras, chamadas de {\em coeficientes\/}
do polin�mio, e $a_k > 0$. 

Seja $k$ uma constante inteira positiva. Dizemos, de uma fun��o $f$
sobre os naturais, que:

  \begin{itemize}
    \item $f$ tem limite (superior) de complexidade polinomial se
      $f(n) = O(n^k)$,
    \item tem limite (inferior) de complexidade exponencial se $f(n) =
      \Omega(k^n)$,
    \item tem limite (superior) de complexidade logar�tmica se $f(n) =
      O((lg n)^k)$.
  \end{itemize}

% $lg^k n$ � abrevia��o de $(lg n)^k$.

$lg$ indica logaritmo na base 2; usamos $log$ para especificar a base,
  de forma que $lg n = log_2 n$. A base de logaritmos em complexidade
  algor�tmica n�o � muito relevante, pois, para quaisquer constantes
  inteiras $a,b,c>0$, temos: $log_b a = \frac{log_c a}{log_c b}$.

$lg lg k$ � abrevia��o de $lg (lg k)$, e $lg$ tem pouca preced�ncia
  (s� se aplica ao pr�ximo termo em uma f�rmula): $lg n + k$ significa
  $(lg n) + k$.

Para quaisquer constantes inteiras $a,b$, se $a>1$ temos:

  \[ \begin{array}{l}
       \llim_{n\rightarrow\!\infty} \frac{n^b}{a^n} = 0 % \label{expxpoli}
            \\
            \llim_{n\rightarrow\!\infty} \frac{lg^b n}{n^a} = 0 % \label{lgxpoli}
     \end{array}
  \]
A primeira equa��o indica que toda fun��o exponencial com uma base
maior que 1 cresce mais rapidamente que qualquer fun��o polinomial.

A segunda equa��o (que vale tamb�m se $a=1$) indica que toda fun��o
logar�tmica cresce mais lentamente que qualquer fun��o polinomial.

%Lembre-se:

%  \[ e^x \begin{array}[t]{l}
%         = \llim_{n \rightarrow \infty} (1 + \frac{x}{n})^n \\
%         = \sum_{i=0}^\infty \frac{x^i}{i!}
%         \end{array}
%  \]


\section{Exerc�cios Resolvidos}

\begin{enumerate}

\item $2^{n+1} \preceq 2^n$ ? 

Resposta: Sim. Pois $2^{n+1} = 2*2^n$. Constantes n�o interferem na
ordem de complexidade (na defini��o de $O$, basta escolher a constante
$c$ adequadamente, neste caso podemos escolher qualquer $c\geq 2$).

\item $2^{2n} \preceq 2^n$ ? 

Resposta: N�o.

Suponha que sim. Dever�amos ter ent�o $c,n_0$ tais que $0\leq 2^{(2n)}
\leq c2^n$, para $n\geq n_0$. Dividindo por $2^n$ --- note:
$2^{2n}=(2^n)^2$ --- obtemos: $2n \leq c$, para todo $n\geq n_0$, o
que � falso.

A fun��o que recebe $n$ e retorna $2^{2n}$ � chamada de duplamente
exponencial.

\item Usa-se $\lfloor x \rfloor$ (o ch�o de $x$) para denotar o menor
  inteiro maior ou igual a $x$, e $\lceil x \rceil$ (o teto de $x$)
  para denotar o maior inteiro menor ou igual a $x$.

  A fun��o $\lfloor lg n \rfloor$ tem limite de complexidade
  polinomial?

Resposta: Sim. 

$\lfloor x \rfloor < x + 1$, e $lg n \preceq n$, e portanto $\lfloor
lg n \rfloor \preceq n$.

\item Cada entrada da tabela abaixo indica, para o par formado por $v
  = f(n)$ e $w = g(n)$ das duas primeiras colunas da tabela, se $f(n)
  = X(g(n))$, para $X$ variando de acordo com o indicado nas colunas
  seguintes (i.e.~para $X = \Theta,O,\Omega,o,\omega$).  Insira {\tt
    '*'} quando a entrada da tabela abaixo for "Sim", e deixe em
  branco caso contr�rio.
% Sempre que inserir {\tt '*'}, justifique (mostre porque sim).

Suponha $k\geq 1$, $a > 0$ e $b > 1$ constantes.

%\newcommand{\sin}{{\it sin\/}}

\[ \begin{array}{|c|c|c|c|c|c|c|}
v        & w        & \Theta & O & \Omega & o & \omega \\ \hline\hline
(lg n)^k & n^a      &         &  &        &   &        \\ \hline
n^k      & b^n      &         &  &        &   &        \\ \hline
\sqrt n  & n^{sin n} &         &  &        &   &        \\ \hline
2^n      & 2^{n/2}   &         &  &        &   &        \\ \hline
n^{lg b}  & b^{lg n}  &         &  &        &   &        \\ \hline
lg (n!)  & lg (n^n) &         &  &        &   &        
    \end{array}
\]

Resposta (baseada em \cite{Minko-Markov-2013}):

\[ \begin{array}{|c|c|c|c|c|c|c|}
    v        & w      & \Theta & O      & \Omega  & o       & \omega\\ \hline
                                                                        \hline
(lg n)^k & n^a        &       & {\tt *} &         & {\tt *} &     \\ \hline
n^k      & b^n        &       & {\tt *} &         & {\tt *} &     \\ \hline
\sqrt n  & n^{\sin\ n} &        &         &         &         &     \\ \hline
2^n      & 2^{n/2}     &       &         & {\tt *} &         & {\tt *} \\ \hline
n^{lg b}  & b^{lg n}    & {\tt *} & {\tt *} & {\tt *} &        &        \\ \hline
lg (n!)  & lg (n^n) &          & {\tt *} &        & {\tt *} &        
    \end{array}
\]

A primeira linha � consequ�ncia de $\llim_{n\rightarrow\!\infty}
\frac{(lg n)^k}{n^a} = 0$, para todo $k$ e todo $a>0$.

A segunda linha � consequ�ncia de $\llim_{n\rightarrow\!\infty}
\frac{n^k}{b^n} = 0$, para todo $a$ e todo $b>1$.

$\sin\ n$ oscila (continuamente) entre 0 e 1, quando $n$ cresce de 0 a
$\infty$. Portanto, $2^{1/2}$ e $2^{\sin\ n}$ n�o se relacionam (com as
rela��es assint�ticas acima).

$\frac{2^{n/2}}{2^n} = \frac{1}{2^{n/2}}$, e portanto $2^n \preceq
2^{n/2}$.

Temos: $\begin{array}[t]{lll}
          lg (n^{lg b}) & = lg b (lg n) & \asymp (lg n) \\
          lg (b^{lg n}) & = lg n (lg b) & = \asymp (lg n) 
        \end{array}$ e, portanto, $n^{lg b} \asymp b^{lg n}$. 

Temos: 

\[ \begin{array}[t]{llllllll}
n!  = n & \times (n-1) & \times (n-2) & \times (n-3) & \times \ldots 
                                      & \times 3 & \times 2 & \times 1\\
n^n = n & \times  n    & \times  n    & \times n     & \times \ldots
                                      & \times n & \times n & \times n
\end{array}
\]

As duas linhas t�m exatamente $n$ termos, e cada termo do lado direito
da primeira � menor que o termo correspondente da segunda linha, para
$n$ suficientemente grande.  Logo, $n! \leq n^n$ para $n$
suficientemente grande.

\item Ordene as seguintes fun��es sobre os naturais por ordem de
  complexidade. Ou seja, ordene as fun��es de $f_1$ a $f_{30}$ de modo
  que $f_i \prec f_{i+1}$ ou $f_i \asymp f_{i+1}$, para $i=1,\ldots,29$.

  As fun��es, aplicadas a $n$, s�o: 

   \[ \begin{array}{|c|c|c|c|c|c|}
 lg (lg n) & (\sqrt{2})^{lg n}  & n^2          & (3/2)^n       & n^3         & lg^* n        \\
 2^{2^n}    & n^{(\frac{1}{lg n})}  & n^n          & lg n          & 2^{lg n}     & (lg n)^{lg n}  \\
 2^n       & 4^{lg n}           & n lg n       & 2^{2^{n+1}}     & n!          & (lg n)!       \\
 (n+1)!    & lg (n!)           & lg (lg^* n)  & 2^{lg * n}     & lg^* (lg n) & 2^{\sqrt{2 lg n}} \\
 n^{lg lg n} & 1                 & \sqrt{lg n}  & n            & n 2^n       & (lg n)^2
      \end{array}
   \]

A nota��o $lg^*$ � definida a seguir.  Considere primeiro que
$f^{(i)}$ denota a fun��o ``f aplicada $i$ vezes'', para $i\geq 0$:

\[ f^{(i)} x = 
    \left\{ \begin{array}{ll}
      x            & se $i=0$ \\
      f(f^{(i-1)} x) & caso contr�rio
    \end{array}\right. 
\]

$lg^*$ � uma fun��o que recebe um argumento $n$ e retorna o menor $i$
tal que $lg^{(i)} n \leq 1$, ou, em outras palavras, retorna quantas
vezes se precisa aplicar $lg$ para obter-se 1 ou menos:

  \[ lg^* n = min \{ i \mid i\geq 0, lg^{(i)} n \leq 1 \} \]

Por exemplo, $\begin{array}[t]{llll} 
lg^* 2          &             & = 1 & (lg^{(0)} 2 = 2, lg^{(1)} 2 = lg (lg^{(0)} 2) = lg 2 = 1) \\
lg^* 3          &             & = 2 & (lg^{(0)} 3 = 3, lg^{(1)} 3 = lg 3, lg^{(2)} 3 = lg (lg 3) = 0.6644\ldots) \\
lg^* 2^2        & = lg^* 4     & = 2 & (lg^{(0)} 4 = 4, lg^{(1)} 4 = lg 4 = 2, lg^{(2)} 4 = lg 2 = 1) \\
lg^* 5          &             & = 3 & (lg^{(0)} 5 = 5, lg^{(1)} 5 = lg 5 = 2.3219\ldots, lg^{(2)} 5 = lg 2.3219\ldots = 1.2153\ldots, \ldots)\\
\ldots          &             &     &       \\
lg^* 2^{2^2}     & = lg^* 16    & = 3 & \\ 
               & = lg^* 17    & = 4 & \\
\ldots         &              &     & \\ 
lg^* 2^{2^{2^2}}  & = lg^* 65536 & = 4 & \\
\ldots         & = lg^* 65537 & = 5 & \\
\ldots         &              &     & \\
lg^* 2^{2^{2^{2^2}}} & = \ldots     & = 5 & \\
\ldots         &              &       
              \end{array}$

Ou seja, $lg^*$ cresce {\em muito\/} lentamente. S� poder�amos
escrever o pr�ximo valor ($ n_6 = 2^{2^{2^{2^{2^2}}}} $) usando
exponencia��o ($ n_5 = 2^{2^{2^{2^2}}} $ tem 19729 d�gitos, mas $n_6$
tem um n�mero de d�gitos extraordin�rio, ``maior do que o n�mero de
�tomos ($\approx 10^{82}$) que se estima existir no universo que
podemos observar'' (i.e.~o universo que se expande at� 90 e poucos
bilh�es de anos-luz; $10^{82}$ � da ordem de (menor que)
$2^{(10/3)\times 82}$, que � muito menor que $2^{65536}$). Note que
$10^3$ � aproximadamente igual (um pouco menor que) $2^{10}$, e
portanto $10^k = 10^{3 \times (k/3)} = (10^3)^{k/3}$, que � portanto
da ordem de $(2^{10})^{k/3} = 2^{10 \times (k/3)} = 2^{(10/3)\times
  k}$.

Solu��o:

\begin{enumerate}[[\textbf{1}]

\item $1 \asymp n^{\frac{1}{lg n}}$

Isso pode ser mostrado tomando $lg$ de $n^{\frac{1}{lg n}}$. Temos que
$lg (n^{\frac{1}{lg n}}) = (\frac{1}{lg n}) \times lg n = 1$ (e
portanto $n^{\frac{1}{lg n}} = 2$).  
Logo, $1 \asymp n^{\frac{1}{lg n}}$.

\item $1 \prec lg (lg^* n)$

Direto. Pois $1 \prec lg (lg^* n)$ decorre de $1 < 2 * lg (lg^* 4)$
--- pela defini��o das rela��es $(\prec)$ e $O$, tomando $n_0 = 4, c = 2$ e usando o
fato de que $lg (lg^* n)$ � monot�nica, i.e.~cresce ou continua igual
quando $n$ cresce. O que � v�lido pois: 
 $1 < 2 * lg (lg^* 4)$ � o mesmo que
 $1 < 2 * lg 2$, ou seja, $1 < 2$.

\item $lg (lg^* n) \prec lg^* n$

Seja $m = lg^* n$. Temos ent�o que provar: $lg m \prec m$.

$lg m \prec m$ � consequ�ncia de $\llim_{n\rightarrow\!\infty} \frac{(lg n)}{n} = 0$.

\HRule
{\em Nota\/}: 
O fato de que esse limite � zero � conhecido, mas pode ser obtido
usando a regra de l'H�pital, como a seguir.

Usando $'$ (diz-se: 'linha') para denotar a derivada de uma fun��o, a
regra de l'H�pital especifica que, para todas as fun��es $f$, $g$
diferenci�veis em um intervalo aberto $I$ exceto possivelmente em um
ponto $k \in I$, se i) $\llim_{x \to k}f(x)=\lim_{x \to k}g(x)=v$,
onde $v = 0$ ou $v = \pm\infty$, ii) $\lim_{x\to
  k}\frac{f '(x)}{g'(x)}$ existe, e iii) $g'(x)\neq 0$ para todo $x\in
I - \{ k\}$, ent�o $\lim_{x\to k}\frac{f(x)}{g(x)} = \lim_{x\to
  c}\frac{f '(x)}{g'(x)}$.

Assim, para todo $k>0$, temos: 
  $\lim_{n\rightarrow\!\infty} \frac{(lg n)}{n^k}$ � igual a 
  $\lim_{n\rightarrow\!\infty} \frac{(\frac{ln n}{ln 2})}{n^k}$ 
que, pela regra de l'H�pital, usando o fato de que 
  $(ln n)' = \frac{1}{n}$ 
e $(n^k)' = k \times n^{k-1}$, 
obtemos 
  $\lim_{n\rightarrow\!\infty} \frac{(\frac{1}{n \times (ln 2) \times k})}{1}$, 
que � igual a 0. Ou seja, para todo $k>0$ temos:
  \begin{equation}  
    \lim_{n\rightarrow\!\infty} \frac{(lg n)}{n^k} = 0 \label{lim-poli-sobre-exp}
  \end{equation} 

Para mostrar que a derivada de $ln$ � a fun��o inversa (i.e.~$(ln n)'= \frac{1}{n}$),
seja: $y = ln x$, ou seja, $e^y = x$; derivando ambos os lados em
rela��o a $x$, temos: $e^y (\frac{dy}{dx}) = 1$, ou seja, 
 $x (\frac{dy}{dx}) = 1$, isto �: $\frac{dy}{dx} = \frac{1}{x}$.

{\em Fim de Nota\/} 
\HRule

\item $lg^* n \asymp lg^* (lg n)$

Veja a varia��o de valores da fun��o $lg^*$ (veja explica��o acima). A
diferen�a entre $lg^* n$ e $lg^* (lg n)$ � $1$, ou seja, $lg^* (lg n)
= (lg^* n) - 1$.

\item $lg^* n \prec 2^{lg^* n}$

Consequ�ncia de $m \prec 2^m$ (fazendo $m = lg^* n$). 

\item $2^{lg^* n} \prec lg (lg n)$

Para poder comparar mais facilmente, eliminamos a exponencia��o
tomando o logaritmo (aplicando $lg$) aos dois lados. Obtemos: $lg^* n
\prec lg (lg (lg n))$. Como $lg^* n$ s� cresce com o n�mero de
pot�ncias de 2 de uma torre de pot�ncias de 2 que expressa o valor de
$n$, e como $lg 2^i = i$, com $n$ a partir de $n_5 = 2^{2^{2^{2^2}}}$,
para o qual $lg^* n_5 = 5$ e $lg (lg (lg n_5)) = 2^{2^2} = 16$,
teremos sempre $lg^* n$ menor que $lg(lg(lg n))$.

\item $lg (lg n) \prec \sqrt{lg n}$

Com $lg n = m$ obtemos $lg m \prec m^{\frac{1}{2}}$, que �
consequ�ncia de (\ref{lim-poli-sobre-exp}).

\item $\sqrt{lg n} \prec lg n$

Com $lg n = m$ obtemos $\sqrt m \prec m$, o que � verdadeiro (na
defini��o de O, basta escolher $n_0 = 1, c = 1$).

\item $lg n \prec (lg n)^2$

Com $lg n = m$ obtemos $m \prec m^2$, o que � verdadeiro (na defini��o
de O, basta escolher $n_0 = 1, c = 1$).

\item $(lg n)^2 \prec 2^{\sqrt{2 lg n}}$

Com $lg n = m$ obtemos $m^2 \prec 2^{\frac{m}{2}}$, i.e.~$4m^2 \prec 2^m$, 
o que � consequ�ncia de $\lim_{n\rightarrow\!\infty} \frac{n^b}{a^n} = 0$, 
para todas as constantes $a$ e $b$ tais que $a>1$. 

\item $2^{\sqrt{2 lg n}} \prec \sqrt{2}^{lg n}$

Temos: $\sqrt{2}^{lg n} = 2^{\frac{lg n}{2}}$. Assim, usando o fato de
que, para todo $k,f,g$, $k^{f(n)} \prec k^{g(n)}$ se e somente se
$f(n) \prec g(n)$, obtemos o resultado desejado se e somente se
$\sqrt{2 lg n} \prec \frac{lg n}{2}$, o que � verdadeiro pois
$\sqrt{lg n} \prec lg n$.

\item $\sqrt{2}^{lg n} \prec n$

Temos: $\sqrt{2}^{lg n}  \prec n$ se e somente se 
       $2^{\frac{lg n}{2}} \prec n$ se e somente se 
       $2^{lg \sqrt{n}}   \prec n$ se e somente se 
       $\sqrt{n}       \prec n$,
o que � verdadeiro. % (na defini��o de $O$, basta escolher $n_0 = 4$, $c = 1$).

\item $2^{lg n} \asymp n$

Aplicando $lg$, obtemos: $lg n \asymp lg n$.

\item $n \prec n lg n$.

Podemos escolher por exemplo $n_0 = 1$, $c=2$ na defini��o de $O$.

\item $n lg n \asymp lg (n!)$

Usando a {\em aproxima��o de Stirling\/}:

  \[ n! = \sqrt{2\pi n} (\frac{n}{e})^n (1 + \Theta(\frac{1}{n})) \]
e aplicando $lg$ a ambos os lados, obtemos: $lg (n!) = lg (\sqrt{2\pi n}) + n lg n - n lg e$, 
que � assintoticamente igual a $n lg n$:

  \begin{equation}
    lg (n!) \asymp n lg n 
    \label{lgn-eq-nlgn}
  \end{equation}

\item $n lg n \prec n^2$

Podemos escolher por exemplo $n_0 = 2$, $c=1$ na defini��o de $O$.

\item $n^2 \asymp 4^{lg n}$

Pois $4^{lg n} = (2^2)^{lg n} = 2^{2^{lg n}} = 2^{2 lg n} = 2^{lg (n^2)} = n^2$. 

\item $n^2 \prec n^3$

Podemos escolher por exemplo $n_0 = 1$, $c=2$ na defini��o de $O$.

\item $n^3 \prec (lg n)!$

Aplicando $lg$ a ambos os lados, obtemos $lg (n^3) \prec lg ((lg n)!)$. 
Com $m = lg n$, obtemos: $3 m \prec lg (m!)$ (pois $lg (n^3) = 3 lg n$). 

Usando (\ref{lgn-eq-nlgn}) obtemos: $3 m \prec m lg m$, que significa
$3 \prec lg m$, que � verdadeiro.

\item $(lg n)! \prec (lg n)^{lg n}$. 

� equivalente a $m!\prec m^m$, com $m = lg n$. O que � verdadeiro, pois:

  \[ \lim \frac{m \times (m-1) \times \ldots 2 \times 1}
               {m \times m \times \ldots m \times m} = 0 
  \]
  
\item $(lg n)^{lg n} \asymp n^{lg (lg n)}$

Vamos usar o fato de que $log_b a^n = n log_b a$, para todo $a,b,n$.

Aplicando $lg$ ao lado esquerdo, obtemos: 

 \[ lg ((lg n)^{lg n}) = lg n \times lg (lg n) \]

Aplicando $lg$ ao lado direito, obtemos:

  \[ lg (n^{lg (lg n)}) = lg (lg n) \times lg n \]

\item $n^{lg (lg n)} \prec (\frac{3}{2})^n$

Aplicando $lg$ ao lado esquerdo, obtemos: $lg n \times lg (lg n)$. 

Aplicando $lg$ ao lado direito, obtemos: $n lg \frac{3}{2}$. 

Temos $n \prec lg (lg n)$ e $lg (lg n) \prec lg n \times lg (lg n)$.

Por transitividade da rela��o $(\prec)$, obtemos: $n \prec lg n \times
lg (lg n)$, e portanto $n lg \frac{3}{2} \prec lg n \times lg (lg n)$.

O resultado � ent�o obtido pelo fato de que:

  \begin{equation}
    \text{Para toda fun��o} f,g, \text{ temos: }
        lg f(n) \prec lg g(n) \text{ se e somente se } f(n) \prec g(n) 
    \label{lgprec}
  \end{equation}

\item $(\frac{3}{2})^n \prec 2^n$

Consequ�ncia de: $\lim_{n\rightarrow\!\infty} \frac{(\frac{3}{2})^n}{2^n} = \lim_{n\rightarrow\!\infty} (\frac{3}{4})^n = 0$.

\item $2^n \prec n 2^n$

Com $m = 2^n$ ($n = lg m$), obtemos: $m \prec m lg m$.

\item $n 2^n \prec n!$

Temos: $2^n \prec n!$ e, por transitividade, uma vez que $n 2^n \prec
2^n$, obtemos $n 2^n \prec n!$.

\item $n! \prec (n+1)!$

Consequ�ncia de: $(n+1)! = (n+1) \times (n!)$.

\item $(n+1)! \asymp n^n$

Aplicando $lg$ a ambos os lados, obtemos, usando (\ref{lgn-eq-nlgn}):
  $(n+1) lg (n+1) \asymp n lg n$, o que � verdadeiro.
Obtemos o resultado usando (\ref{lgprec}).

\item $n^n \prec 2^{2^n}$

Aplicando $lg$ a ambos os lados, obtemos, usando (\ref{lgn-eq-nlgn}):
  $n lg n \prec lg (2^n)$, o que � verdadeiro.
Obtemos o resultado usando (\ref{lgprec}).

\item $2^{2^n} \prec 2^{2^{n+1}}$

Temos: $2^{n+1} = 2 \times 2^n$ e portanto $2^{2^{n+1}} = 2^{2\times
  {2^n}} = 2^{2^n} \times 2^{2^n}$.

\end{description}

%\end{enumerate}

\section{Exerc�cios}

\begin{enumerate}

\item  xxx

\end{enumerate}



<<<<<<< HEAD
%%!TEX encoding = ISO-8859-1
\chapter{\colorbox{cyan}{Estruturas de Dados B�sicas}}
\label{estruturas-de-dados-basicas}

Este cap�tulo aborda listas e �rvores, suas representa��es em um computador e
opera��es b�sicas sobre essas estruturas de dados.

\section{Listas}
\label{listas}

Uma lista � uma estrutura de dados comumente usada em computa��o e pode ser definida recursivamente como a seguir.  Uma lista de elementos de
determinado tipo \inh{t} � ou i) {\em vazia} ou ii) constitu�da de um elemento de tipo \inh{t} e um uma lista de elementos do mesmo tipo \inh{t} (denominada {\em cauda} ou resto da lista).

Em uma linguagem como Haskell, que prov� suporte � defini��o de tipos
de dados recursivos, o tipo lista pode ser definido como:

\begin{center}
\begin{tabular}{l}
\begin{hask}{List,Nil,Cons}
data List a = Nil | Cons a (List a)
\end{hask}
\end{tabular}
\end{center}

O tipo de dado \inh{\List a} � um tipo recursivo, sendo \inh{Nil} e \inh{Cons} os construtores de valores desse tipo. Al�m disso, \inh{\List a} � um tipo {\em polim�rfico}: a vari�vel de tipo \inh{a} pode ser instanciada para um tipo \inh{t} qualquer, permitindo assim a defini��o de listas com elementos de desse tipo \inh{t}. Por exemplo, \inh{(Cons 1 (Cons 2 (Cons 3 Nil)))} � uma lista de tipo \inh{List Int} e \inh{(Cons True (Cons False Nil))} tem tipo \inh{List Bool}. 

A linguagem Haskell prov� uma nota��o especial para a cria��o de valores de tipo lista: \inh{[]}  � usado em lugar de \inh{Nil}, e o construtor infixado \inh{(:)}, associativo � direita, em lugar de \inh{Cons}. Por exemplo, a lista \inh{Cons 1 (Cons 2 (Cons 3 Nil))} seria representada como \inh{(1:2:3:[]}. 
Al�m disso, uma lista pode ser representada simplesmente escrevendo-se os seus elementos da lista entre colchetes, separados por v�rgulas. Ou seja, a lista \inh{\inh{(1:2:3:[]}} pode ser escrita, mais simplesmente, como \inh{[1,2,3]}. 

Em linguagens como \C, uma lista pode ser representada por meio de {\em registros} (tamb�m chamados de  "estruturas", em \C) e {\em ponteiros\/} (ou apontadores), tal como ilustrado no exemplo a seguir, que define uma lista de elementos de tipo \ina{int}. Linguagens como \C, que n�o prov�m suporte para polimorfismo, requerem a defini��o de tipos lista distintos para cada tipo particular de elementos. 

\begin{center}
\begin{tabular}{l}
\begin{alg}{ListaDeInteiros}
struct ListaDeInteiros
   int elem; 
   struct ListaDeInteiros *r;
\end{alg}
\end{tabular}
\end{center}

Uma defini��o de registro, em \C, � introduzida pela palavra-chave \ina{struct}, seguida do nome do registro -- neste caso, \ina{ListaDeInteiros}. O registro possui dois campos: um campo de tipo \ina{int} e nome \ina{elem}, e um campo de nome \ina{r}, que � um ponteiro para valores do pr�prio tipo \ina{ListaDeInteiros}. 
 
A manipula��o de valores de tipo lista em \C\ � bem mais trabalhosa. A falta de suporte para defini��o e uso de tipos recursivos e polim�rficos torna a programa��o mais dif�cil e demorada e o c�digo menos leg�vel e mais sujeito a repeti��es e a ocorr�ncias de erros. Por exemplo, para criar um valor de tipo \ina{ListaDeInteiros}, com os elementos \inh{1,2,3},  � necess�rio c�digo como o seguinte:
\index{\ina{ListaDeInteiros}}

\begin{center}
\begin{tabular}{l}
\begin{alg}{ListaDeInteiros}
struct ListaDeInteiros *p =
   malloc(sizeof(struct ListaDeInteiros))
   p->elem = 1;       p->r = malloc(sizeof(struct ListaDeInteiros))
   p->r->elem = 2;    p->r->r = malloc(sizeof(struct ListaDeInteiros))
   p->r->r->elem = 3; p->r->r->r = NULL
\end{alg}
\end{tabular}
\end{center}

Em \Haskell, o acesso a um valor \inh{v}, em uma lista \inh{x}, requer acesso a
todos os elementos anteriores a \inh{v} em \inh{x}. De fato, a representa��o interna de listas definidas em \Haskell � feita por meio de ponteiros, mas a manipula��o de ponteiros � gerada automaticamente pelo compilador da linguagem, de acordo com o c�digo do programa, em vez de ser feita diretamente pelo programador.

Uma maneira alternativa de representar listas � por meio de {\em arranjos}, 
especialmente em uma linguagem (como \C) que n�o prov� suporte a
manipula��o de valores de estruturas de dados recursivas. Utilizando essa forma de representa��o, a lista fica limitada a um n�mero m�ximo de elementos, j� que a defi��o de uma arranjo requer que o n�mero de elementos do mesmo seja especificado a priori.

\subsection{Pesquisa}
\label{pesquisa-em-lista}

Em computa��o, {\em pesquisar\/} em geral significa determinar se um
dado elemento est� presente ou n�o em uma estrutura de dados. As
subse��es seguintes tratam de opera��es de pesquisa, inser��o e
remo��o de elementos de listas, de acordo com a forma com que listas
s�o representadas.

Em ambos os casos apresentados abaixo, a pesquisa em uma lista de $n$
elementos tem complexidade $O(n)$ no pior caso, pois envolve, possivelmente, compara��o com cada elemento da lista.

\subsubsection{Vers�o funcional}

A fun��o \inh{elem}, que determina se um dado valor � elemento de uma lista dada, pode ser definida como a seguir:
\index{\inh{elem}}

\begin{center}
\begin{tabular}{l}
\begin{hask}{elem}
elem :: Eq a => a -> a -> Bool
elem a []    = False
elem a (b:x) = (a == b) || elem a x
\end{hask}
\end{tabular}
\end{center}

O tipo de \inh{elem}, em \Haskell, � um tipo {\em polim�rfico restrito\/}: a restri��o ({\em constraint\/}) (\inh{Eq a}) indica que a vari�vel de tipo \inh{a}
n�o pode ser instanciada para qualquer tipo, mas apenas para um tipo
que � membro da classe de tipos \inh{Eq}, ou seja, no caso, apenas para um
tipo para o qual exista definida uma opera��o de igualdade \inh{(==)}, para
valores desse tipo. � um erro de tipo chamar a fun��o \inh{elem} com um argumento de um tipo para o qual n�o � definida compara��o de igualdade.

\HRule
{\em Nota\/}: 

A fun��o \inh{elem}, de fato, faz parte do m�dulo \Prelude, importado
automaticamente por todos os m�dulos de programas \Haskell, sem necessidade de comando ou cl�usula expl�cita de importa��o. A defini��o de \inh{elem} no \Prelude\ � diferente da apresentada acima, e usa outras fun��es tamb�m definidas no \Prelude, como \inh{mmap} e \inh{foldr}, que s�o ferramentas importantes para defini��o de outras fun��es em Haskell. A defini��o de \inh{elem} contida no \Prelude\ � apresentada a seguir:

\begin{center}
\begin{tabular}{l}
\begin{hask}{foldr,mmap,and,or,any,all,elem}
foldr  :: (a -> b -> b) -> b -> [a] -> b
foldr f z []    =  z
foldr f z (a:x) =  f a (foldr f z x) 

mmap :: (a -> b) -> [a] -> [b]
mmap �{\tt \_}�  []   = []
mmap f (a:x) = f a : mmap f x 

and, or :: [Bool] -> Bool
and =  foldr (&&) True
or   =  foldr (||) False 

any, all :: (a -> Bool) -> [a] -> Bool
any p =  or . map p
all p =  and . map p 

elem :: (Eq a) => a -> [a] -> Bool
elem a = any (== a)
\end{hask}
\end{tabular}
\end{center}

\HRule

\subsubsection{Vers�o imperativa}

A vers�o imperativa de \ina{elem} definida a seguir recebe como argumento o valor a ser pesquisado, denotado pelo par�metro \ina{a}, juntamente com um apontador para uma lista (de tipo \ina{ListaDeInteiros}), denotado pelo par�metro \ina{l}. O algoritmo retorna retorna um apontador para o elemento da lista \ina{l} que � igual a \ina{a}, caso o argumento esteja presente na lista, e retorna \ina{NULL} em caso contr�rio.
\index{\ina{elem}}

\begin{center}
\begin{tabular}{l}
\begin{alg}{elem}
elem (a, l) 
   while (l != NULL && l->elem != a) l = l->r
   return l
\end{alg}
\end{tabular}
\end{center}

\subsection{Inser��o}
\label{insercao-em-lista}

Inserir um elemento no in�cio de uma lista � uma opera��o de
complexidade $O(1)$.

\subsubsection{Vers�o funcional}

A inser��o de um elemento no in�cio da lista � feita simplesmente, por meio do construtor de lista \inh{(:)}, ou seja:
\index{\inh{insert}}

\begin{center}
\begin{hask}{insert}
insert = (:)
\end{hask}
\end{center}

\subsubsection{Vers�o imperativa}

Na vers�o imperativa, � alocado um novo registro, de tipo \ina{ListaDeInteiros},
para armazenar o novo valor a ser inserido na lista, sendo retornada a lista resultante dessa opera��o.
\index{\ina{insert}}

\begin{center}
\begin{tabular}{l}
\begin{alg}{insert}
insert (a, l)
   p = malloc(sizeof(struct ListaDeInteiros))
   p->elem = a
   p->r = l
   return p
\end{alg}
\end{tabular}
\end{center}

\subsection{Remo��o}
\label{remocao-de-lista}

Remover um elemento de uma lista � uma opera��o de complexidade $O(n)$
no pior caso, pois � necess�rio procurar o elemento a ser removido.

A Se��o \ref{lista-duplamente-encadeada} redefine o tipo de lista
encadeada para uma vers�o de listas duplamente encadeadas, e reescreve
as fun��es \ina{elem} e \ina{insert}, para definir \ina{delete} por meio de uma
chamada � fun��o \ina{elem}, seguida de chamada a \ina{insert}.

\subsubsection{Vers�o funcional}

A vers�o funcional cria uma nova lista, que n�o tem o elemento
passado como par�metro:
\index{\inh{delete}}

\begin{center}
\begin{tabular}{l}
\begin{hask}{delete}
delete :: Eq a => a -> [a] -> [a]
delete �{\tt \_}� [] = []
delete a (b:x)
   | a == b    = x
   | otherwise = b:delete a x
\end{hask}
\end{tabular}
\end{center}

\subsubsection{Vers�o imperativa}

A vers�o imperativa de \ina{delete}, apresentada a seguir, n�o cria uma nova lista: usa um ponteiro -- \ina{prev} -- para percorrer a lista at� encontrar o elemento a ser removido e, quando este � encontrado, modifica a estrutura de encadeamento da lista, removendo este elemento. 

\begin{center}
\begin{tabular}{l}
\begin{alg}{delete}
delete (a, l) 
   struct ListaDeInteiros *prev = NULL;  
   *p = l
   while (p != NULL && p->elem != a)
      prev = p
      p = p->r
      if (prev == NULL) 
         return l->r
      else { prev->r = p->r; return l }
\end{alg}
\end{tabular}
\end{center}
             

\subsection{Lista duplamente encadeada}
\label{lista-duplamente-encadeada}

\ldots \ldots .....

\subsection{Pilha}
\label{pilha}

Uma estrutura de dados \ina{Pilha} caracteriza-se pelo fato de que as opera��es de inser��o, acesso e remo��o de elementos s�o feitas em apenas
um de seus lados (ou extremidades). Essa pol�tica de uso � comumente  chamada LIFO (do ingl�s, {\em last-in first-out\/}: o �ltimo a ser inserido � o primeiro a ser removido da pilha. A implementa��o de uma \ina{Pilha} inclui as  com as opera��es: i) criar pilha vazia, ii) empilhar elemento, iii) desempilhar elemento, iv) obter elemento do topo da pilha, e v) testar se pilha est� vazia. 

\subsubsection{Vers�o funcional}
\index{\inh{Pilha}}
Em \Haskell, a implementa��o de \inh{Pilha} � obtida diretamente das opera��es definidas sobre listas, isto �:

\begin{center}
\begin{tabular}{l}
\begin{hask}{vazia,empilhar,desempilhar,topo,estaVazia}
vazia = []
empilhar = (:)
desempilhar (�{\tt \_}�:x) = x
topo (a:�{\tt \_}�) = a
estaVazia = null
\end{hask}
\end{tabular}
\end{center}
                  
Em geral, vamos procurar simplificar o c�digo de nossos programas,
n�o tratando casos de erro, na maioria das vezes, por motivos did�ticos. Entretanto, em programas completos n�o podemos esquecer de tratar todos os casos poss�veis para os dados de entrada. Um m�dulo em Haskell que trata
todos esses casos poss�veis para os dados de entrada das opera��es
acima � mostrado na Figura \ref{fig-Pilha}.

O m�dulo \inh{Pilha} implementa o que � chamado em computa��o de um {\em tipo abstrato de dados\/}, que � um tipo com opera��es de cria��o,
modifica��o e consulta sobre valores desse tipo. Por exemplo, \inh{vazia} � uma opera��o de cria��o (nesse caso, a �nica);
\inh{empilhar} e \inh{desempilhar} s�o opera��es de modifica��o; \inh{topo} e \inh{estaVazia} s�o opera��es de consulta. 

Em uma defini��o de um tipo abstrato de dados, a defini��o de tipo e das opera��es (para cria��o, modifica��o e consulta) sobre valores
do tipo devem ser contidas em um mesmo trecho de programa (em geral,
um m�dulo), e a representa��o usada n�o � ``vis�vel'' para quem usa valores do tipo. Ou seja, um tipo abstrato � constitu�do de um tipo, munido de um conjunto de opera��es sobre valores desse tipo. Qualquer outra opera��o sobre valores do tipo apenas pode ser implementada por meio dessas opera��es previamente definidas.

Para definir um tipo abstrato \inh{Pilha}, em \Haskell, usamos uma defini��o de um novo tipo, introduzida pela palavra-chave \inh{newtype}. O tipo \inh{Pilha} possui um �nico construtor de valores \inh{mkPilha} (de mesmo nome do construtor de tipo). Esse mecanismo � usado para ocultar a representa��o do tipo abstrato: o construtor de  que n�o � exportado pelo m�dulo em que o tipo � definido: o construtor de valores \inh{mkPilha} n�o � exportado, apenas o construtor de tipos \inh{Pilha}. Para melhor legibilidade, definimos tamb�m o tipo de cada uma das fun��es. Veja Figura \ref{fig-Pilha}.
\end{document}

\begin{figure}

\begin{center}
\begin{tabular}
\begin{hask}{Pilha,vazia,empilhar,desempilhar,topo,estaVazia}
module Pilha (Pilha, vazia, empilhar, desempilhar, topo, estaVazia) where
   newtype Pilha a = mkPilha [a] 

vazia :: Pilha $a
vazia = mkPilha []

empilhar :: a -> Pilha a -> Pilha a
empilhar e (mkPilha p) = Pilha (e:p)

desempilhar :: Pilha a -> Pilha a
desempilhar (mkPilha []) = error "Fun��o desempilhar chamada com pilha vazia"
desempilhar (mkPilha (�{\tt \_}�:p)) = mkPilha p

topo:: Pilha a -> a
topo (mkPilha [])   = error "Fun��o topo chamada com pilha vazia"
topo (mkPilha (e:�{\tt \_}�)) = e 

estaVazia:: Pilha a -> Bool
estaVazia (mkPilha $p$) = null p
\end{hask}
\end{tabular}
\end{center}

\label{fig-Pilha}
\caption{Tipo abstrato \Pilha\ em Haskell}
\end{figure}



\subsubsection{Vers�o imperativa}

Considerando \pilha\ como um registro com campos \topo\ e \elems,
sendo \elems\ um arranjo de $n$ elementos --- indexado de $0$ a {\tt
  $n$-1} --- e \topo\ uma vari�vel inteira que indica o �ndice do
�ltimo elemento inserido, as opera��es em uma pilha podem ser
implementadas como a seguir (desconsiderando casos de erro:
desempilhar de uma pilha vazia e empilhar em uma pilha cheia).

\newcommand{\nome}{{\it nome\/}}

O comando \with\ serve para tornar os nomes de campos de um registro
vis�veis: \with\ $r$ evita que se tenha que prefixar os nomes dos
campos do registro $r$ com o registro (\nome\ pode ser usado em vez de
{\tt $r$.\nome}).

\progb{
  \vazia\ (\pilha) \{ \pilha.\topo\ = -1 \}\\ \ \hspace*{.2cm} \\
  \empilhar\ (\eelem, \pilha) \\
      \hspace*{.2cm} \with\ \pilha\  \\
        \hspace*{1cm} \topo\ $\leftarrow$ \topo\ + 1\\
        \hspace*{1cm} \elems[\topo] $\leftarrow$ \eelem\\ \ \hspace*{.2cm} \\
  \desempilhar\ (\pilha) \\ 
      \hspace*{.2cm} \with\ \pilha\ \{ \topo\ $\leftarrow$ \topo\ - 1 \} \\ \ \hspace*{.2cm} \\
  \topo\ (\pilha) \{ \with\ \pilha\ \{ \return\ \elems[\topo] \} \}\\ \ \hspace*{.2cm} \\ 
  \estaVazia\ (\pilha) \{ \return\ \pilha.\topo\ == -1 \}
 }

\subsection{Fila}
\label{fila}

Em uma {\em fila} a inser��o � feita de um lado e a remo��o � feita do
outro lado da estrutura de dados. Isso implica em uma pol�tica algumas
vezes chamada de FIFO ({\em first-in first-out\/}: o primeiro a ser
inserido � o primeiro a ser removido da fila.

Uma fila, com opera��es de i) criar fila vazia, ii) entrar (inserir
elemento) na fila, iii) sair (tirar elemento) da fila, iv) obter
elemento do in�cio da fila, e v) testar se fila est� vazia, pode ser
implementada como a seguir.

\subsubsection{Vers�o funcional}

\newcommand{\frente}{{\it frente\/}}
\newcommand{\tras}{{\it tr�s\/}}
\newcommand{\fila}{{\it fila\/}}

N�o � eficiente fazer acesso ao �ltimo elemento de uma lista em
Haskell. A implementa��o padr�o de filas por meio de listas usa assim
duas listas, \frente\ e \tras. Elementos entram na lista \tras\ e saem
na lista \frente.

A fun��o \fila\ � usada para garantir o invariante de que se
\frente\ est� vazia, ent�o \tras\ est� vazia (e portanto a fila est�
vazia).

\newcommand{\reverse}{{\it reverse\/}}
\newcommand{\sair}{{\it sair\/}}
\newcommand{\entrar}{{\it entrar\/}}
\newcommand{\inicio}{{\it in�cio\/}}

\progb{\vazia\ = ([],[])\\ \ \hspace*{.2cm} \\
      \entrar\ $e$ (\frente,\tras) = \fila\ (\frente, $e$ : \tras)\\ \ \hspace*{.2cm} \\
      \fila\ ([], \tras) = (\reverse\ \tras, [])\\
      \fila\ $f$        = $f$\\ \ \hspace*{.2cm} \\
      \sair\ ($e$:\frente, \tras) = (\frente,\tras)\\ \ \hspace*{.2cm} \\
      \estaVazia\ (\frente,\_) = \null\ \frente\\ \ \hspace*{.2cm} \\
      \inicio\ ($e$:\frente,\_) = $e$
     }

\subsubsection{Vers�o imperativa}
\label{fila-imperativa}

\newcommand{\fim}{{\it fim\/}}

Considerando \fila\ como um registro com campos \inicio, \fim\ e
\elems, sendo \elems\ um arranjo de $n$ elementos --- indexado de $0$
a {\tt $n$-1}.  Os �ndices do primeiro e do �ltimo elementos inseridos
s�o armazenados respectivamente nas vari�veis \inicio\ e \fim. 

A fila est� vazia quando {\tt \inicio\ == \fim}. A fila est� cheia
quando {\tt \inicio\ == \fim\ + 1}, isto �, a fila � circular: o
�ndice {\tt 0} segue o �ndice {\tt $n$-1}. 

As opera��es em uma fila podem ser implementadas como a seguir,
desconsiderando casos de erro: sair de uma fila vazia e entrar em uma
fila cheia. O operador {\tt \%} � usado para retornar o resto da
divis�o do primeiro operando pelo segundo.

\progb{
  \vazia\ (\fila) \{ \with\ \fila\ \{ \inicio\ = \fim\ = 0 \} \} \\ \ \hspace*{.2cm} \\
  \entrar\ (\eelem, \fila) \\
      \hspace*{.2cm} \with\ \fila\  \\
          \hspace*{1cm} \elems[\fim] $\leftarrow$ \eelem\\
          \hspace*{1cm} \fim\ $\leftarrow$ (\fim\ + 1) \% $n$\\ \ \hspace*{.2cm} \\
  \sair\ (\fila) \\
      \hspace*{.2cm} \with\ \fila\ \{ \inicio\ $\leftarrow$ (\inicio\ + 1) \% $n$ \}\\ \ \hspace*{.2cm} \\
  \inicio\ (\fila) \{ \with\ \fila\ \{ \return\ \elems[\inicio] \} \}\\ \ \hspace*{.2cm} \\
  \estaVazia\ (\pilha) \{ \with\ \fila\ \{ \return\ \inicio\ == \fim\ \} \} 
 }

%!TEX encoding = ISO-8859-1
\section{�rvores}
\label{sec:arvores}
\index{arvores}

Uma �rvore pode ser vista como uma estrutura de dados recursiva, definida como sendo ou i) vazia (uma folha) ou ii) um nodo contendo um elemento e um certo n�mero de ramos (ou nodos), que cont�m sub-�rvores:

\begin{center}
\begin{tabular}{l}
\begin{hask}{Tree,Leaf,Node}
data Tree a = Leaf | Node a [Tree a]
\end{hask}
\end{tabular}
\end{center}

Na defini��o acima, o construtor \inh{Leaf} constr�i uma �rvore vazia, ou folha, e o construtor \inh{Node} constr�i uma �rvore contendo uma informa��o de tipo \inh{a} e um lista de sub�rvores.

Outra poss�vel defini��o considera que a informa��o � armazenada nas folhas, em vez de nos nodos internos:

\begin{center}
\begin{tabular}{l}
\begin{hask}{Tree,Leaf,Node}
data Tree' a = Leaf' a | Node' [Tree' a]
\end{hask}
\end{tabular}
\end{center}

Uma �rvore com exatamente duas sub-�rvores (possivelmente vazias) � chamada de �rvore bin�ria e pode ser definida como:

\begin{center}
\begin{tabular}{l}
\begin{hask}{BTree,BLeaf,BNode}
 data BTree a = BLeaf a | BNode (BTree a) (BTree a)
\end{hask}
\end{tabular}
\end{center}

%---- acho que esse par�grafo deve ser eliminado
%---- � prematuro falar em grafo e nem � apresentado um desenho
%---- talvez isso possa ser deixado como nota
%
%Uma �rvore pode tamb�m ser definida como um {\em grafo} conexo e  ac�clico. Um {\em grafo\/} � simplesmente um conjunto de v�rtices e de arestas entre esses v�rtices. Dizemos que dois v�rtices $a$ e $b$ do grafo s�o {\em adjacentes\/} se est�o conectados por uma aresta -- usualmente representada como um par$(a,b)$. Um grafo � {\em conexo\/} se todo v�rice � adjacente a algum outro. Um {\em caminho\/} de um v�rtice $a$ a um v�rtice $b$ � uma sequ�ncia de v�rtices adjacentes, tendo $a$ como primeiro e $b$ como �ltimo v�rtice. Um {\em ciclo\/} � um caminho que inicia e termina no mesmo v�rtice, n�o repetindo nenhuma aresta. Um grafo � {\em ac�clico\/} se n�o cont�m nenhum ciclo. Um grafo ac�clico, mas n�o for conexo, � uma floresta, isto �, um conjunto de �rvores.
%
%-----------------------------------------------------------

Em linguagens que prov�em suporte ao uso de ponteiros, mas n�o �
defini��o e manipula��o direta de tipos recursivos, a representa��o de
�rvores bin�rias pode ser feita com o uso de ponteiro como mostra o
exemplo a seguir:

\begin{center}
\begin{tabular}{l}
\begin{alg}{ArvoreBinariaDeInteiros}
struct ArvoreBinariaDeInteiros
     int elem
     struct ArvoreBinariaDeInteiros *esq, dir 
\end{alg}
\end{tabular}
\end{center}

Os campos \ina{esq} e \ina{dir} de um nodo s�o ponteiros, possivelmente nulos,
para sub-�rvores.

Para �rvores n�o bin�rias, pode ser usada a seguinte representa��o, que
podemos chamar de {\em representa��o com primog�nito-irm�o-e-pai}:

\begin{center}
\begin{tabular}{l}
\begin{alg}{ArvoreDeInteiros}
struct ArvoreDeInteiros
   int elem;
   struct ArvoreDeInteiros *primogenito, irmao, pai 
\end{alg}
\end{tabular}
\end{center}

A representa��o da �rvore da Figura \ref{fig:Arv1} � mostrada na Figura \ref{fig:Rep-arv1}.

\begin{figure}

xxxx
\caption{�rvore exemplo}
\label{fig:Arv1}
\end{figure}

\begin{figure}

yyyyy
\caption{Representa��o da �rvore exemplo por valor do tipo \ina{ArvoreDeInteiros}}
\label{fig:Rep-arv1}
\end{figure}
\end{document}


=======
%!TEX encoding = ISO-8859-1
\chapter{An�lise da efici�ncia de algoritmos}
\label{analise-eficiencia-de-algoritmos}

Esse cap�tulo apresenta um roteiro para an�lise da efici�ncia de
algoritmos, juntamente com exemplos simples de problemas e suas solu��es,
usando esse roteiro.

Al�m da efici�ncia, algoritmos podem ser analisados quanto a
facilidade de mostrar ou provar corre��o, simplicidade e
generalidade. Ao contr�rio da an�lise da efici�ncia, simplicidade e facilidade de
mostrar corre��o s�o crit�rios bastante subjetivos. � bastante dif�cil
estabelecer m�tricas para tais crit�rios. Generalidade, por sua vez,
pode ser medida pelo tamanho do dom�nio da entrada do problema
resolvido, mas h� situa��es em que o desenvolvimento de um algoritmo
mais geral � desnecess�rio (pouco vantajoso) ou dif�cil, e tal
dificuldade ou necessidade � dif�cil de ser medida precisamente.

Em geral, o projeto de algortimos envolve a ado��o de solu��es que
favorecem um aspecto em detrimento de outro, e um aspecto que costuma
ser bastante influente � o tempo dispon�vel para desenvolvimento do
programa. O desenvolvimento de algoritmos {\em �timos\/} � uma quest�o
relativa ao {\em problema\/} que est� sendo resolvido e, mesmo
restringindo ao aspecto de efici�ncia, para muitos problemas saber
dizer qual � o algoritmo �timo � dif�cil e muitas vezes n�o tem uma
resposta conhecida. 

%Vamos falar mais sobre esse assunto na se��o \ref{sec:P-vs-NP}.

\HRule
{\em Nota\/}: 

A prova de corre��o de programas � uma �rea da ci�ncia da computa��o
que est� em franca evolu��o, atualmente. O desenvolvimento de teorias
de tipos \cite{Sorensen98lectureson}, baseadas nos chamados ``tipos
dependentes'' \cite{Bove:2009:DTW,Nederpelt-Geuvers-2014}, tem
evolu�do bastante. Esse desenvolvimento tem estimulado trabalhos com
os chamados ``assistentes de prova''
\cite{Geuvers2009:Proof-assistants}. Esses programas e linguagens, no
entanto, ainda requerem bastante treinamento e parecem ainda estar em
processo de evolu��o, antes que possam ser mais amplamente
usados. Atualmente, a corre��o da vasta maioria dos programas usados
na pr�tica n�o � demonstrada, mas sujeita a testes. Provas de corre��o
e t�cnicas de teste de programas n�o fazem parte do escopo deste
livro; no entanto, vamos usar provas de indu��o e defini��o de
invariantes para mostrar informalmente a corre��o de programas.

\HRule

Como usualmente, vamos em geral omitir a {\em valida��o dos dados de
  entrada}, isto �, n�o verificar se os dados de entrada realmente
est�o dentro dos limites estabelecidos no enunciado de um problema. Em
programas usados na pr�tica, essa verifica��o deve ser inclu�da mas,
em geral, essa valida��o n�o envolve nenhum aspecto mais relevante
para a tarefa de programa��o, envolvendo apenas a inclus�o de testes
para emiss�o de mensagens de erro no caso em que os dados de entrada
n�o satisfa�am a esses testes.

As se��es a seguir apresentam alguns exemplos de problemas e
algoritmos simples para solu��o desses problemas, seguidos da an�lise
de efici�ncia desses algoritmos. O roteiro para an�lise da efici�ncia
� o seguinte:

\begin{enumerate}

\item Determinar a vari�vel ($n$) que representa o tamanho dos dados de
  entrada.

\item Identificar opera��es que s�o relevantes para determinar a 
  efici�ncia do programa durante a execu��o.

\item Expressar o n�mero de vezes que essas opera��es s�o executadas,
  em fun��o de $n$, chamada de express�o-determinante-da-efici�ncia.

\item Resolver ou simplificar a express�o-determinante-da-efici�ncia.

\end{enumerate}

No caso de um programa recursivo, a
express�o-determinante-da-efici�ncia �, em geral, tamb�m definida
recursivamente. Isto �, o tempo de execu��o do algoritmo para uma
entrada de tamanho $n$ -- $T(n)$ -- � escrito em termos de $T(n-1)$,
ou em termos de $T(k)$, para algum $k<n$. Dizemos, nesse caso, que
$T(n)$ � definido por uma {\em rela��o de recorr�ncia}.

\begin{quotation}
  {\em Uma rela��o de recorr�ncia � uma defini��o recursiva para a
    qual busca-se uma solu��o n�o recursiva que a simplifique, que
    especifica a mesma rela��o.\/}
\end{quotation}

Existem in�meras equa��es recursivas para as quais n�o podemos
encontrar uma solu��o n�o recursiva (de fato, isto acontece com a
maioria das equa��es recursivas). Entretanto, para os problemas com os
quais vamos lidar a seguir, as equa��es recursivas que expressam o
tempos de execu��o de algoritmos para solu��o desses problemas podem
ser resolvidas de maneira bastante simples.

No caso de um programa n�o recursivo, a
express�o-determinante-da-efici�ncia � em geral um somat�rio, que
muitas vezes tamb�m pode ser simplificado, usando propriedades de
somat�rios como as seguintes, onde $a,b,c$ s�o constantes:

  \[ \begin{array}{l}
       \sum_{i=a}^{b} cm_i = c \sum_{i=a}^{b} m_i\\
       \sum_{i=a}^{b} (m_i + n_i) = \sum_{i=a}^{b} m_i + \sum_{i=a}^{b} n_i\\
       \sum_{i=a}^{b} (m_i - n_i) = \sum_{i=a}^{b} m_i - \sum_{i=a}^{b} n_i
     \end{array}
  \]
O Ap�ndice \ref{Somatorios} apresenta e discute propridades de
somat�rios.

As se��es seguintes apresentam exemplo de problemas simples e suas
solu��es, para os quais a efici�ncia � analisada usando o roteiro
acima. Um m�todo simples de substitui��o para obten��o de uma f�rmula
geral que simplifique a rela��o de recorr�ncia � apresentado por meio
desses exemplos. Vamos chamar o m�todo de {\em
  substitur-para-generalizar\/}.

\section{N�mero de D�gitos}
\label{numero-de-digitos}

O problema � determinar o n�mero de d�gitos de um n�mero em uma dada
base usada para representa��o desse n�mero. O n�mero e a base s�o
dados de entrada.

\subsection{Vers�o funcional}

A vers�o funcional � apresentada em Haskell a seguir:

\begin{center}
\begin{tabular}{l}
\begin{hask}{numDigs}{\divisao}
numDigs x b 
  | x < b     = 1
  | otherwise = 1 + numDigs (x `div` b)
\end{hask}
\end{tabular}
\end{center}

A vari�vel que representa o tamanho dos dados de entrada � igual a
$n$.  A varia��o do tempo de execu��o $T(n)$ � dada por (considerando
como $k$ uma constante igual ao tempo gasto pela opera��o de somar 1 a
um valor qualquer mais o tempo gasto pela opera��o de comparar o
argumento \inh{x} com \inh{b}):

 \[ \begin{array}{lll}
       T(n) & = 0                      & \text{ se } n < \inh{b}\\
       T(n) & = T(n `div` \inh{b}) + k & \text{ caso contr�rio}
    \end{array}
 \]
Vamos considerar que $n$ � uma pot�ncia de \inh{b} --- isto �, $n =
\inh{b}^i$, para algum $i\geq 0$. Essa considera��o � baseada na {\em
  regra-de-crescimento-suave}, descrita sucintamente na nota no final
dessa se��o (para mais detalhes, veja o Ap�ndice
\ref{relacoes-de-recorrencia}).

Para $i\geq b$, obtemos:  
 \[ \begin{array}{ll}
       T(\inh{b}^i) & = T(\inh{b}^{i-1}) + k \\
              & = T(\inh{b}^{i-2}) + (2 \times k) \\
              & \ldots
    \end{array}
 \]
Para $n=\inh{b}^i$, obtemos $T(\inh{b}^i) = T(\inh{b}^0) + (i\times k) = i\times k$.  

Portanto, $T(n) = log_{\inh{b}} (i\times k)$ e portanto $T(n) \asymp lg n$.

\HRule
{\em Nota\/}: 

Uma fun��o $f$ � {\em eventualmente n�o-decrescente\/} se existe $n_0$
tal que $f(n_2) \geq f(n_1)$, para todo $n_2 > n_1 \geq n_0$ no
dom�nio de $f$.

Uma fun��o $f$ sobre os naturais {\em cresce suavemente\/} se �
eventualmente n�o-decrescente e $f(2n) \asymp f(n)$.

Fun��es logar�tmicas, polinomiais e combina��es lineares de logaritmos
e polin�mios s�o todas fun��es que crescem suavemente.  Por exemplo, a
fun��o $f$ definida por $f(n) = n lg n$ cresce suavemente, pois:
$f(2n) = 2n lg (2n) = 2n (lg 2 + lg n) = (2 lg 2)n + 2n lg n \asymp n lg n$. 

Fun��es exponenciais com base maior que 1 e fatoriais n�o crescem
suavemente. Por exemplo, a fun��o $f$ definida por $f(n) = 2^n$ n�o
cresce suavemente, pois $f(2n) = 2^{2n} = 4^n \not\asymp 2^n$.
  
N�o � dif�cil mostrar que, para toda fun��o $f$ que cresce suavemente
e para todo $k\geq 2$, temos: $f(kn) \asymp f(n)$. 

A regra-de-crescimento-suave especifica: se $f$ � uma fun��o
eventualmente n�o-decrescente, $g$ cresce suavemente e $f(n)\asymp g(n)$ 
para valores de $n$ que s�o pot�ncias de $b$, onde $b\geq 2$,
ent�o $f(n) \asymp g(n)$.

\HRule

\subsection{Vers�o imperativa}

A vers�o imperativa � similar, usando um comando de repeti��o em vez
de recurs�o:

\begin{center}
\begin{tabular}{l}
\begin{alg}{numDigs}{\divisao}
numDigs (n,b)
  numD = 0
  while (n > b) 
       numD = numD + 1
       n = n / b
\end{alg}
\end{tabular}
\end{center}

A express�o-determinante-da-efici�ncia � igual a $m \times \Theta(1)$,
onde $m$ � o n�mero de vezes que o comando de repeti��o � executado e
$\Theta(1)$ expressa o tempo gasto nos comandos internos ao comando de
repeti��o. Como a vari�vel \ina{n} recebe, a cada repeti��o, o valor
do quociente da divis�o do valor de \ina{n} (anterior � atribui��o)
por \ina{b}, obtemos: 
 $T(\ina{n}) \asymp m \asymp log_{\ina{b}} \ina{n} \asymp lg \ina{n}$.

Note que $T(\ina{n}) \asymp lg \ina{n}$ para qualquer base \ina{b}.

Note tamb�m que $T(\ina{n})$ (e o n�mero de repeti��es no \while)
aumenta logaritmicamente com um aumento (linear) no {\em valor\/} de
\ina{n}, mas aumenta linearmente com um aumento no n�mero de d�gitos
de \ina{n} (o valor de \ina{n} aumenta exponencialmente com um
aumento no n�mero de d�gitos de \ina{n}).

\section{Maior Elemento}
\label{maior-elemento}

Considere o problema de encontrar o maior elemento de uma lista. 

A vers�o funcional apresentada abaixo simplesmente usa \foldl'
(definida em \inh{Data.List}):

\begin{center}
\begin{tabular}{l}
\begin{hask}{maxElem}{\decremento}
maxElem :: Ord a => [a] -> a
maxElem (a:x) = foldl' max a x
\end{hask}
\end{tabular}
\end{center}

A fun��o \inh{max}, definida no prel�dio de Haskell, retorna o maior
entre dois valores, passados como par�metros:

\begin{center}
\begin{tabular}{l}
\begin{hask}{max}{\definicao}
max :: Ord a => a -> a -> a
max a b 
  | a <= b    = b
  | otherwise = a
\end{hask}
\end{tabular}
\end{center}
 
A fun��o \inh{foldl} � definida no prel�dio de Haskell como a seguir:

\begin{center}
\begin{tabular}{l}
\begin{hask}{foldl}{\decremento}
foldl f z []    = z
foldl f z (a:x) = foldl f (f z a) 
\end{hask}
\end{tabular}
\end{center}

A fun��o \inh{foldl}, aplicada a uma fun��o bin�ria \inh{f}, um valor
inicial \inh{z} e uma lista, ``reduz'' (em geral, mas mais
precisamente transforma) a lista usando a fun��o \inh{f} da esquerda
para a direita (da� o nome \inh{foldl}: o {\it l\/} � de {\em
  {\underline{l}eft\/}}, em portugu�s {\em esquerda\/}):

  \[  \inh{foldl f z} [e_1, e_2, \ldots, e_n]
      \inh{ == } 
         (\ldots ((\inh{z `f`} e_1 ) \inh{`f`} e_2) \inh{`f`}\ldots) \inh{`f`}
                   e_n
  \]

A fun��o \inh{foldl'} se comporta de modo similar a \inh{foldl}, mas �
``menos pregui�osa'': \inh{foldl' f} for�a a avalia��o de \inh{f}, de
modo que, \inh{z `f`} $e_1$ � avaliado antes que a express�o $(\inh{z
  `f`} e_1) \inh{`f`} e_2$ seja formada, e assim sucessivamente. Com
\inh{foldl}, toda a express�o $(\ldots ((\inh{z `f`} e_1) \inh{`f`}
e_2) \inh{`f`}\ldots) \inh{`f`} e_n$ � constru�da antes da avalia��o
de $\inh{z `f`} e_1$. O uso de \inh{foldl'} � adequado quando a fun��o
\inh{f} � estrita (ou seja, quando \inh{f} n�o � pregui�osa), ou,
ainda, quando a avalia��o de $\inh{z `f`} e_1$ requer a avalia��o de
$e_1$.

\HRule
{\em Nota\/}: 

Uma fun��o $f$ � dita estrita se (escrevendo sucintamente) $f\: \bot =
\bot$, ou seja: quando o resultado de aplicar $f$ a um argumento que
fica em ciclo infinito faz com que a chamada a $f$ fique em ciclo
infinito. O valor $\bot$ � usado para indicar ``ciclo infinito'', e
tamb�m ocorr�ncia de erro devido a recursos, em quantidade finita,
serem consumidos para avalia��o, durante a execu��o. 

Em Haskell, ao contr�rio da grande maioria das linguagens de
programa��o, a estrat�gia de avalia��o de express�es � ``pregui�osa''
(em ingl�s, ``lazy''). Isso significa que o argumento para uma fun��o
n�o � avaliado quando a express�o � chamada, mas simplesmente
substitu�do pelo par�metro no corpo da fun��o, para posterior
avalia��o, se necess�rio. Al�m disso, se for necess�ria, a avalia��o
do argumento s� � feita uma �nica vez, na estrat�gia de avalia��o
pregui�osa; nas outras vezes em que o argumento for usado, � usado o
valor resultante da avalia��o feita na primeira vez.

Na maioria das linguagens de programa��o, a estrat�gia de avalia��o de
express�es � ``gulosa'' (em ingl�s, ``eager''). Nessa estrat�gia, o
argumento � avaliado antes de uma chamada � fun��o. Essa estrat�gia
faz com que todas as fun��es sejam estritas.

As diferen�as resultantes do uso de estrat�gias de avalia��o
pregui�osa e gulosa est�o fora do escopo deste livro. Para os
programas que vamos analisar neste livro, a an�lise de efici�ncia de
programas com estrat�gia de avalia��o pregui�osa n�o difere da an�lise
com a estrat�gia de avalia��o gulosa. 

\HRule
{\em Nota\/}: 

Este livro prov� uma introdu��o a an�lise do tempo de execu��o de
programas. A complexidade de tempo de programas usando avalia��o
pregui�osa � sempre igual ou menor (sendo $\preceq$ a ordem de
compara��o) que a complexidade usando avalia��o estrita. Isso ocorre
porque a avalia��o de cada express�o pode n�o ser realizada e, se for
realizada, a avalia��o � feita apenas uma vez. A diferen�a mais
significativa entre as duas estrat�gias � relativa n�o ao tempo mas ao
espa�o gasto durante a execu��o de programas: a estrat�gia de
avalia��o pregui�osa pode requerer mais espa�o durante a execu��o,
justamente para armazenamento de informa��es para avalia��o de
express�es que n�o foram, mas eventualmente poder�o ter que ser,
avaliadas. Isso pode causar diminui��o da constante usada na
complexidade de tempo do programa, e pode fazer com que que o espa�o
necess�rio para avalia��o do programa seja maior do que o que foi
alocado para essa execu��o (pode ocorrer, por exemplo, o que � chamado
de ``estouro de pilha''). Por outro lado, h� casos em que ocorre o
contr�rio. A estrat�gia gulosa permite o uso e manipula��o de
estruturas de dados de tamanho maior do que o que seria poss�vel com a
estrat�gia de avalia��o gulosa, e mesmo estruturaa de dados de tamanho
ilimitado. Por exemplo, considere a fun��o \inh{f n = take n [1..]}.
Apesar de usar uma lista de tamanho ilimitado, ela tem o mesmo
comportamento, em termos de significado e em termos do tempo de
execu��o gasto para sua execu��o, do que a fun��o \inh{g n = [1..n]}.

\HRule

No entanto, vale observar que o uso de \inh{foldl'} neste exemplo n�o
altera a complexidade de tempo, mas pode afetar bastante o espa�o
necess�rio e a constante de proporcionalidade da fun��o que expressa o
tempo de execu��o.

\HRule

A rela��o de recorr�ncia � 

 \begin{equation}
    T(n) = T(n-1) + k 
    \label{recorrencia1}
 \end{equation}
onde $n$ � o tamanho (n�mero de elementos) da lista e $k$ � uma
constante que expressa o tempo de execu��o da aplica��o da fun��o
$\max$ a dois valores inteiros. No caso base, temos $T(0) = 0$.

Essa rela��o de recorr�ncia tem solu��o f�cil. Vamos mostrar sua
solu��o pelo {\em
  m�todo-de-substitui��o-para-elimina��o-da-recurs�o\/} ilustrado a
seguir. 

Temos: $T(n-1) = T(n-2) + k$. Substituindo $T(n-1)$ na rela��o de
recorr�ncia (\ref{recorrencia1}), obtemos: 
  $T(n) = (T(n-2) + k) + k = T(n-2) + 2\times k$. 
� f�cil ver que, para todo $i=1,\ldots,n$, temos:
$T(n) = T(n-i) + i\times k$. Para $i=n$, temos: $T(n) = n\times k$ e,
portanto, $T(n) = \Theta(n)$.

Note que n�o � preciso o m�todo acima (de substituir-para-generalizar)
para concluir que $T(n) = n\times k$; basta raciocinar sobre a
defini��o de $T$:

    \[ \begin{array}[t]{ll}
         T(n) = 0          & \text{ se } n = 0\\
         T(n) = T(n-1) + k & \text{ caso contr�rio}
       \end{array}
    \]

A express�o que define $T(n)$ � igual a 0 quando $n$ � igual a 0 e
aumenta de $k$ qundo $n$ aumenta de 1; ou seja, � uma defini��o de
$n\times k$.

O que chamamos de express�o-determinante-da-efici�ncia pode ser
expresso por $T(n-1) + k$, ou (depois de resolvida a rela��o de
recorr�ncia) por $n\times k$.

A complexidade �, assim, a mesma da pesquisa sequencial em uma lista:
$O(n)$ no pior caso. Informalmente, o racioc�nio pode ser similar ao
seguinte: o tempo de execu��o � $O(n)$ pois o custo das opera��es
realizadas em chamada recursiva � constante ($O(1)$) e o n�mero de
chamadas recursivas � $O(n)$ (cresce linearmente com $n$).

\subsection{Vers�o imperativa}

A vers�o imperativa considera por simplicidade um n�mero m�ximo de
elementos $n > 0$ para determinar o maior dos elementos armazenados em
um arranjo (com �ndices de :

\begin{center}
\begin{tabular}{l}
\begin{alg}{maxElem}{\decremento}
maxElem (A) 
  max �$\from$� A[0]
  for i �$\from$� 1 to n-1 
     if A[i] > max
        max �$\from$� A[i]
  return max
\end{alg}
\end{tabular}
\end{center}

A express�o-determinante-da-efici�ncia � igual a $\ina{n} \times
\Theta(1)$, onde $\Theta(1)$ expressa o tempo gasto para compara��o e
atribui��o (nos comandos de sele��o e atribui��o internos ao comando
de repeti��o). Obtemos: $T(\ina{n}) \asymp \sum_{i=1}{\ina{n}} \Theta(1)
\asymp \ina{n}$.

\section{Unicidade}
\label{unicidade}

Considere o problema de verificar se todos os elementos de uma dada
sequ�ncia de elementos s�o distintos. 

\subsection{Vers�o funcional}

A vers�o funcional � apresentada a seguir:

\begin{center}
\begin{tabular}{l}
\begin{hask}{allUnique,chkUnique}{\decremento}
allUnique :: Eq a => [a] -> Bool
allUnique = fst . foldr chkUnique (True,[])

chkUnique :: Eq a => a -> (Bool,[a]) -> (Bool,[a])
chkUnique a (True,x) 
 | a `elem` x   = (False,undefined)
 | otherwise    = (True, a:x)
chkUnique _ r   = r
\end{hask}
\end{tabular}
\end{center}

A fun��o \inh{chkUnique} usa um par $(b,x)$, onde $b$ indica unicidade
(de todos os elementos testados at� agora), e $x$ representa a lista
dos elementos j� testados. A unicidade de cada elemento $a$ da lista �
testada com rela��o � lista dos j� testados $x$: se $a$ n�o pertence a
$x$, ent�o a unicidade � preservada e $a$ � adicionado a $x$, caso
contr�rio n�o h� unicidade e a verifica��o pode ser interrompida.

A fun��o \inh{allUnique} percorre a lista para produzir o resultado
desejado, usando \inh{chkUnique}.

A rela��o de recorr�ncia � 

 \[ T(n) = T(n-1) + f(n) + k \]
onde \inh{n} � o tamanho (n�mero de elementos) da lista, \inh{f} � a
fun��o de complexidade de tempo de \inh{chkUnique} e $k$ � uma
constante que expressa o tempo de execu��o de \inh{fst}.  Temos:
$\inh{f}(\inh{n}) = \inh{n} + k'$, onde $k'$ � uma constante que
expressa o tempo de execu��o \inh{(:) a x}. Obtemos ent�o:

  \[ T(\inh{n}) = T(\inh{n}-i) + i \times (\inh{n} + k + k') \]
Com $i=\inh{n}$, obtemos: 

  \[ T(\inh{n}) = T(0) + \inh{n}^2 + \Theta(\inh{n}) \]
Como $T(0)$ � constante, obtemos: 

  \[ T(\inh{n}) \asymp \inh{n}^2 \]

\subsection{Vers�o imperativa}

Na vers�o imperativa por simplicidade vamos considerar que a sequ�ncia
est� armazenada em um arranjo de tamanho \ina{n}. O programa �
mostrado a seguir:

\begin{center}
\begin{tabular}{l}
\begin{alg}{allUnique}{\decremento}
allUnique(A) 
  for i �$\from$� 0 to n-2 do
    for j �$\from$� i+1 to n-1 do
        if A[i] == A[j] return false
\end{alg}
\end{tabular}
\end{center}

Temos: $T(\ina{n}) = \sum_{i=0}^{\ina{n}-2} \sum_{j=i+1}^{\ina{n}-1} k$,
onde $k$ � o custo de compara��o de dois elementos do arranjo.

Simplificando, obtemos: 

  \[ \begin{array}{lll} 
      T(\ina{n}) & = & \sum_{i=0}^{\ina{n}-2} (k \times (\ina{n} - i - 1)) \\
                 & = & (\ina{n}-1) + (\ina{n}-2) + \ldots + 1 \\
                 & = & (\ina{n}-1) \times \ina{n} / 2
% didn't use \frac 'cause hevea doesn't like \ina{n} inside \frac... :-(
     \end{array}
  \]
Ou seja, $T(\ina{n}) \asymp \ina{n}^2$.

%!TEX encoding = ISO-8859-1
\section{Multiplica��o de Matrizes}
\label{sec:multiplicacao-de-matrizes}

Considere o problema de calcular o produto de duas matrizes quadradas.

Por defini��o, cada elemento $C[i,\,j]$ de duas matrizes $n\times \!n$ $A$
e $B$ � igual a: 

  \[ \begin{array}{llll}
      A[i,0]   & \times & B[0   & , j] + \ldots + \\
      A[i,k]   & \times & B[k   & , j] + \ldots + \\
      A[i,n-1] & \times & B[n-1 & , j] 
     \end{array}
  \]
para $0\leq i,\,j \leq n-1$.

\subsection{Vers�o funcional}
\label{sec:multMatriz-fun}

A motiva��o fundamental para a exist�ncia de arranjos � a propriedade
de prover acesso eficiente ($O(1)$) a seus elementos. Eles espelham
mem�rias RAM (em ingl�s ``random access memories'': mem�rias de acesso
direto). O acesso a elementos de arranjos � feito via indexa��o. Dado
um arranjo \ina{A}, o acesso a cada \ina{i}-�simo elemento � feito
usando \ina{A} e \ina{i} (em \C\ e \Pascal, a nota��o � \ina{A[i]}, em
\Haskell, \inh{A!i}). Uma mem�ria RAM pode ser considerada um arranjo,
sendo o �ndice um endere�o.

A acesso eficiente ($O(1)$) decorre do fato de os elementos serem
armazenados em posi��es cont�guas; dessa forma, o acesso ao elemento
pode ser feito apenas de acordo com o valor do �ndice; n�o � preciso
acesso a outras posi��es do arranjo.

Arranjos em Haskell (2010) fazem parte da biblioteca
\inh{Data.Array}. Essa biblioteca prov� suporte a arranjos imut�veis
(n�o � provido suporte para a a��o de modificar valores contidos em
vari�veis de tipo arranjo). Existem fun��es de modifica��o de
elementos de arranjos na biblioteca, mas tais fun��es retornam novos
arranjos em vez de modificar o arranjo original.

Existem bibliotecas com suporte para arranjos mut�veis, que prov�em
implementa��es de custo $O(1)$ para modifica��o de elementos de
arranjos. A implementa��o de avalia��o de express�es com estrat�gia de
avalia��o tardia requer o uso de ponteiros para valores, em tempo de
execu��o, que complica a implementa��o eficiente de indexa��o de
arranjos, e impossibilita a implementa��o simples de indexa��o de
arranjos feita em linguagens imperativas, basicamente porque elementos
de arranjos s�o ponteiros que podem ou n�o apontar para um valor j�
avaliado. A indexa��o feita em arranjos mut�veis e nos quais os
elementos n�o s�o ponteiros mas os pr�prios valores, devem usar
estrat�gia de avalia��o gulosa e deve usar c�digo mon�dico (c�digo que
usa m�nadas). O assunto de programa��o funcional usando m�nadas e,
portanto, o uso dessas bibliotecas de arranjos mut�veis n�o v�o ser
abordados neste livro.

Em Haskell valores usados como �ndices de arranjo devem ter tipo que �
inst�ncia da classe \inh{Ix}, que � subclasse de \inh{Ord} e define os
nomes \inh{range}, \inh{index} e \inh{inRange} (para uma introdu��o ao
mecanismo de classes de tipos em Haskell, veja o Ap�ndice
\ref{ap:Haskell}, se��o \ref{sec:Classes-de-tipos}):

\begin{center}
\begin{tabular}{l}
\begin{hask}{Ix}{White}
class  (Ord a) => Ix a  where
    range       :: (a,a) -> [a]
    index       :: (a,a) -> a -> Int
    inRange     :: (a,a) -> a -> Bool
\end{hask}
\end{tabular}
\end{center}

Existem, em \inh{Data.Array}, declara��es de inst�ncias (veja se��o
\ref{sec:Classes-de-tipos} para os tipos \inh{Int}, \inh{Integer},
\inh{Char}, \inh{Bool} e para tuplas, triplas, qu�druplas e
qu�ntuplas. Note que em Haskell tuplas podem ser �ndices de matrizes e
outros arranjos multi-dimensionais; inteiros e outros tipos
primitivos, como caracteres, podem ser �ndices de arranjos
uni-dimensionais, tamb�m chamados de {\em vetores\/}.

Por exemplo, um arranjo com limites \inh{((1,1),(10,20))} � um �ndice
com 200 elementos ($10 \times 20$ elementos), e �ndices \inh{(1,1),
  \ldots, (1,10), (2,1), \ldots, (2,20), \ldots, (10,1), \ldots,
  (10,20)}.

A cria��o de um arranjo em Haskell � feita com a fun��o: 

\begin{center}
\begin{tabular}{l}
\begin{hask}{array}{White}
array :: (Ix a) => (a,a) -> [(a,b)] -> Array a b
\end{hask}
\end{tabular}
\end{center}

O primeiro par�metro de \inh{array} especifica o {\em limite
  inferior\/} e {\em limite superior\/} do arranjo, e o segundo uma
lista de pares �ndice-valor, chamada {\em lista geradora do arranjo}.
Por exemplo:

\begin{center}
\begin{tabular}{l}
\begin{hask}{array}{White}
quadrados =  array (1,100) [(i, i*i) | i <- [1..100]]
\end{hask}
\end{tabular}
\end{center}

A vari�vel \inh{quadrados} denota um arranjo que associa a cada �ndice
$i$, de 1 a 100, o quadrado de $i$ (usando defini��o de lista por
gera��o; veja se��o \ref{sec:definicao-de-lista-por-geracao-e-filtragem}).

O operador \inh{!} denota a opera��o fundamental em arranjos, de
indexa��o. Por exemplo:

\begin{center}
\begin{tabular}{l}
\begin{hask}{!}{White}
quadrados!7 
\end{hask}
\end{tabular}
\end{center}
� igual a 49.

O arranjo � igual a \inh{undefined} se algum �ndice na lista geradora
n�o for um �ndice v�lido (i.e.~n�o estiver entre os limites inferior e
superior) do arranjo. Se o �ndice for repetido em um dos pares da
lista geradora, o valor nesse �ndice � igual a \inh{undefined}.

A fun��o \inh{range} recebe um par (limte inferior, limite superior) e
retorna a lista dos �ndices entre esses limites, nesta ordem. Por
exemplo, \inh{range (0,4)} retorna 4, e \inh{range ((0,0),(1,2))}
retorna \inh{[(0,0), (0,1), (0,2), (1,0), (1,1), (1,2)]}.

A fun��o \inh{inRange} determina se um �ndice (segundo argumento) est�
entre os limites (inferior e superior), especificados pelo primeiro
argumento. 

A fun��o \inh{index} \ldots

Um arranjo � estrito no primeiro argumento (limites) e nos �ndices da
lista geradora, mas n�o estrito nos valores dos elementos do arranjo.
Isso permite definir, por exemplo:

\begin{center}
\begin{tabular}{l}
\begin{hask}{fatorial,fibs}{White}
fatorial n = fatArr 
  where fatArr = array (k,n) ((1,1) : [(i, i * fatArr!(i-1)) | i <- [2..n]]) 

fibs n  = arrFib
  where arrFib = array (0,n) ([(0, 1), (1, 1)] ++ 
                              [(i, arrFib !(i-2) + arrFib!(i-1)) | i <- [2..n]]) 
\end{hask}
\end{tabular}
\end{center}

A fun��o \inh{bounds}, definida em \inh{Data.Array}, retorna os
limites inicial e final do arranjo. 
Por exemplo, \inh{bounds quadrados} � igual a \inh{(1,100)}.


A implementa��o de multiplica��o de duas matrizes \inh{m1} e \inh{m2},
mostrada abaixo, constr�i o arranjo que cont�m a multiplica��o
(\inh{m}) de \inh{m1} por \inh{m2} incrementalmente (linha a linha,
elemento a elemento): o $j$-�simo elemento da $i$-�sima linha � igual
ao produto interno da $i$-�sima linha de \inh{m1} pela $j$-�sima
coluna de \inh{m2}, para $i$ variando de 1 at� o n�mero de linhas de
\inh{m} (igual ao n�mero de linhas de \inh{m1}) e $j$ variando de 1
at� o n�mero de colunas de \inh{m} (igual ao n�mero de colunas de
\inh{m2}).  Devido a essa caracter�stica da multiplica��o de matrizes
--- de ser baseada nos produtos das linhas da 1\prima\ matriz
(\inh{m1}) pelas colunas da 2\prima\ (\inh{m2}) --- representamos
\inh{m2} como uma matriz de colunas (a indexa��o de \inh{m2} fornece
uma coluna da matriz).

� inclu�da fun��o \inh{transpoe} para transposi��o de matrizes
(convers�o de representa��o de uma matriz de um arranjo de linhas em
um arranjo de colunas). Para multiplicar matrizes usando
\inh{multMatriz} e considerando a segunda matriz (\inh{m2}) como uma
matriz de linhas, deve-se usar: \inh{multMatriz m1 (transpoe m2)}.

A modifica��o incremental de arranjos � feita pelo operador \inh{(//)
  :: (Ix a) => Array a b -> [(a,b)] -> Array a b}. � usado abaixo o
caso mais simples de atualiza��o, em que o segundo argumento de
\inh{(//)} --- a chamada lista de associa��es --- tem um �nico
elemento. A modifica��o de um arranjo \inh{a} para que o �ndice
\inh{i} passe a conter um valor \inh{v} � dada por \inh{a // [(i,
    v)]}. Como no caso da fun��o \inh{array}, os �ndices na lista de
associa��o devem ser �nicos, para que o valor em cada �ndice da lista
de associa��o seja definido.

\begin{center}
\begin{tabular}{l}
\begin{hask} {multMatriz}{\decremento}

module MultMatriz where

import Data.List  (foldl')
import Data.Array (Array,array,(!),bounds,(//))

type MatrizLin = Array Integer Lin -- MatrizLin = arranjo de linhas 
type MatrizCol = Array Integer Col -- MatrizCol = arranjo de colunas
type Lin       = Array Integer Integer
type Col       = Array Integer Integer
type NumLin    = Integer
type NumCol    = Integer
  
multMatriz :: MatrizLin -> MatrizCol -> MatrizLin
-- Pre: 1) N�mero de colunas de m1 = n�mero de linhas de m2 (i.e. numCols1 = numLins2)
--      2) �ndices de arranjos come�am em 1 (pequena simplifica��o).
multMatriz m1 m2
  | numCols1 == numLins2 = foldl' (addLin m1 m2 numCols1 numCols) m0 [1..numLins]
  | otherwise            = error ("Multiplica��o de matrizes m1 m2 requer: " ++ 
                                  "n�mero de colunas de m1 = n�mero de linhas de m2")
  where
    (_,numLins1) = bounds m1
    (_,numCols2) = bounds m2 -- m2 :: MatrixCol : numCols2 = no. colunas de m2 (n�o de linhas)
    numLins      = numLins1
    numCols      = numCols2
    (_,numCols1) = bounds (m1 ! 1)
    (_,numLins2) = bounds (m2 ! 1)
    m0           = array (1,numLins) [] -- matriz inicial vazia 
                                        -- linhas adicionadas, uma a uma, por addLin

somaLin :: MatrizLin -> MatrizCol -> NumCol -> NumCol -> MatrizLin -> NumLin -> MatrizLin
-- Adiciona m!i. Cada elemento j de m!i = produto interno de m1!i pela coluna j de m2.
somaLin m1 m2 numColsm1 numCols m i = m // [(i,mi)]
  where mi = mkLin i (m1!i) m2 numColsm1 numCols 

geraLin:: NumLin -> Lin -> MatrizCol -> NumCol -> NumCol -> Lin
-- m!i = linha formada pelo produtos internos de m1!i por m2!j, para j = [1..numCols]
geraLin i m1i m2 numColsm1 numCols = array (1,numCols) [(j, prodInterno m1i j (m2!j) numColsm1) | j <- [1..numCols]]
      
prodInterno :: Lin -> NumCol -> Col -> NumCol -> Integer
prodInterno m1i j m2j n = sum [(m1i!j) * (m2j!j) | j <- [1..n]]

transpoe :: MatrizLin -> MatrizCol
-- Pre: m ! i � a i-�sima linha de m, indexada de 1 a n.
-- Pos: transpose m ! i � linha de transpose m tal que m!i!j = transpose m!j!i (j=1,..,n).
transpoe m = array (1,numCols) [(i, array (1,numLins) [(j, m!j!i) | j <- [1..numLins]]) | i <- [1..numCols]]
  where
    (_,numLins) = bounds m
    (_,numCols) = bounds (m!1)
\end{hask}
\end{tabular}
\end{center}

O tamanho dos dados de entrada � dado por $n$, n�mero de elementos em
uma linha ou coluna da matriz quadrada. A rela��o de recorr�ncia �
dada por (temos $n = \inh{numLins} = \inh{numCols}$):

 \[ T(n) = T(n-1) + f_1(n) \]
onde $f_1$ � a fun��o de complexidade de tempo de \inh{addLin}. Temos:
$f_1(n) \asymp f_2(n)$, onde $f_2$ � a fun��o de complexidade de tempo
de \inh{mkLin} (considerando que $f_2(n)$ � maior assintoticamente que
a soma dos tempos de execu��o de \inh{(//)} e \inh{(!)}). Temos:
$f_2(n) = n f_3(n)$, onde $f_3$ � a fun��o de complexidade de tempo de
\inh{innerProd}. � f�cil ver que $f_3(n) \asymp n$.

Obtemos ent�o: $T(n) = T(n-1) + n^2$. Podemos resolver a rela��o de
recorr�ncia usando o m�todo de substitur-para-generalizar; obtemos
$T(n) = T(n-i) + i \times n^2$ e, com $i=n$ e com $T(0)$ constante,
obtemos:

    \[ T(n) \asymp n^3 \]

\subsection{Vers�o imperativa}

No pseudo-c�digo abaixo supomos que: 

  \begin{itemize}
    \item matrizes podem ser usadas como em matem�tica (e acima, e em
      Pascal), ou seja, $A[i,\,j]$ representa o elemento na $j$-�sima
      coluna da $i$-�sima linha da matriz;

    \item uma matriz pode ser retornada como resultado de uma
      fun��o;

    \item os tipos das matrizes $A$ e $B$ s�o matrizes quadradas
      $n\times n$, supondo que esses tipos existem, no pseudo-c�digo
      (e supondo que n�o � preciso definir o tipo de par�metros de
      fun��es);

    \item \ina{alocaMatrizQuadrada (n)} (n�o inclu�da no
      pseudo-c�digo; veja c�digo \C\ a seguir) aloca uma matriz
      quadrada $n\times n$ de inteiros e inicializa todos os elementos
      com \ina{0}.
  \end{itemize}

\begin{center}
\begin{tabular}{l}
\begin{alg}{multMatriz}{\decremento}
multMatriz(A,B) 
  C = alocaMatrizQuadrada (n)
  for i = 0 to n-1 do
    for j = 0 to n-1 do
        for k = 0 to n-1 do
           C[i,j] = C[i,j] + A[i,k] * B[k,j]
  return C
\end{alg}
\end{tabular}
\end{center}

Em uma linguagem como \C, matrizes s�o representadas por arranjos
cujos elementos s�o arranjos (um arranjo de inteiros � representado
por um ponteiro para um inteiro, o primeiro inteiro do arranjo). A
nota��o usada em matem�tica para denotar o elemento na $j$-�sima
coluna da $i$-�sima linha da matriz, $A[i,\,j]$, � denotado por
\ina{A[i][j]}. � preciso primeiro alocar espa�o para os arranjos (isso
� feito usando a fun��o \ina{malloc} (veja o programa abaixo). A
fun��o \ina{malloc} aloca um certo n�mero de unidades de mem�ria e
retorna um ponteiro para o in�cio da �rea alocada.

A fun��o de multiplica��o de matrizes, mostrada a seguir, recebe um
inteiro \ina{n}, duas matrizes \ina{A} e \ina{B} quadradas \ina{n} por
\ina{n} (representadas por meio de ponteiros) e retorna um ponteiro,
que representa a matriz computada pela multiplica��o de \ina{A} por
\ina{B}.

\begin{center}
\begin{tabular}{l}
\begin{alg}{multMatriz}{\decremento}

int** geraMatrizQuad (int n) {
  int i,j; 
  int **p = malloc (n * sizeof(int*)), *q;
  for (i=0; i<n; i++) {
    p[i] = malloc (n * sizeof(int));
    q = p[i];
    for (j=0; j<n; j++) q[j] = 0;
  }
  return p;
}

int** multMatriz (int **A, int **B, int n) {
  int i, j, k;
  int **C = geraMatrizQuad(n);
  for (i=0; i<n; i++) 
    for (j=0; j<n; j++)
      for (k=0; k<n; k++)
        C[i][j] += A[i][k] * B[k][j];
  return C;
}
\end{alg}
\end{tabular}
\end{center}

O tamanho dos dados de entrada � dado por $n$, n�mero de elementos em
uma linha ou coluna da matriz quadrada. As opera��es de atribui��o,
soma e multiplica��o de inteiros, feitas repetidamente no comando
interno ao comando de repeti��o, determinam o tempo de execu��o gasto
pelo programa. Cada uma das atribui��es, incluindo a multiplica��o e
adi��o, tem complexidade $\Theta(1)$. A varia��o do tempo de execu��o
com $n$ � dada por:

  \[ T(n) = \sum_{i=0}^{n-1} \sum_{i=0}^{n-1} \sum_{i=0}^{n-1} \Theta(1) \]
Temos: \[ \begin{array}[t]{lll}
            \sum_{i=0}^{n-1} \Theta(1) = \Theta(n) & 
            \sum_{i=0}^{n-1} \Theta(n) = \Theta(n^2) & 
            \sum_{i=0}^{n-1} \Theta(n^2) = \Theta(n^3) 
          \end{array}
       \]
e portanto $T(n) \asymp n^3$. 



%!TEX encoding = ISO-8859-1
\section{N�meros de Fibonacci}
\label{sec:numeros-de-fibonacci}

Considere a sequ�ncia de n�meros naturais, conhecida como sequ�ncia de
n�meros de Fibonacci: 
  
  \[ 0,1,1,2,3,5,8,13,21,\ldots \]
que pode ser definida pela relac�o de recorr�ncia 

  \[ F(n) = F(n-1) + F(n-2) \]
e pelas condi��es iniciais: $F(0) = 0$, $F(1) = 1$.
 
O problema � determinar o $n$-�simo n�mero da sequ�ncia de n�meros de
Fibonacci, para dado $n$.

\HRule
{\em Nota\/}: 

Curiosidades e hist�ria sobre a sequ�ncia de Fibonacci\ldots ?

\HRule

\subsection{Vers�o Funcional}

Uma solu��o simples � baseada na indexa��o de um arranjo (\inh{fibs n ! n}), 
onde a defini��o de \inh{fibs}, mostrada na se��o
\ref{sec:multMatriz-fun}, � apresentada novamente a seguir:

\begin{center}
\begin{tabular}{l}
\begin{hask}{fatorial,fibs,fib1}{\decremento}
fib1 :: Integer -> Integer
fib1 n = fibs n ! n 
fibs n  = arrFib
  where arrFib = array (0,n) ([(0, 1), (1, 1)] ++ 
                              [(i, arrFib!(i-2) + arrFib!(i-1)) | i <- [2..n]]) 
\end{hask}
\end{tabular}
\end{center}

Contudo, � desnecess�rio usar um arranjo para calcular um dado n�mero
da sequ�ncia de n�meros de Fibonacci. O uso de um arranjo � adequado
se for desejado usar (por exemplo, imprimir) todos ou v�rios n�meros
da sequ�ncia. Nosso problema, no entanto, pode ser resolvido com o
armazenamento de apenas os dois �ltimos n�meros da sequ�ncia. A fun��o
\inh{fib} abaixo faz isso, armazenando passo a passo apenas os dois
�ltimos n�meros da sequ�ncia at� que o n�mero desejado ($n$) seja
alcan�ado:

\begin{center}
\begin{tabular}{l}
\begin{hask}{fib}{\decremento}
fib:: Integer -> Integer
fib 0 = 0
fib 1 = 1
fib n = step (2,(0,1))
  where
    step (k,(fibk_2,fibk_1))
      | k == n    = fibk_2 + fibk_1
      | otherwise = step (k+1,(fibk_1,fibk_2+fibk_1))
\end{hask}
\end{tabular}
\end{center}

A vers�o seguinte, desnecessariamente ineficiente, � obtida usando a
pr�pria defini��o da sequ�ncia de n�meros de Fibonacci:

\begin{center}
\begin{tabular}{l}
\begin{hask}{fatorial,fibs}{\decremento}
fibI 0 = 0
fibI 1 = 1
fibI n = fibI (n-1) + fibI(n-2) 
\end{hask}
\end{tabular}
\end{center}

O tamanho dos dados de entrada � determinado pelo valor de $n$ (�ndice
da sequ�ncia de n�meros de Fibonacci para o qual se deseja determinar
o valor, nesta sequ�ncia).

No caso de \inh{fib1}, as opera��es que determinam a efici�ncia do
programa durante a execu��o est�o relacionas � cria��o do arranjo
\inh{arrFib} (que, como veremos a seguir, � quadr�tica; a opera��o de
indexa��o, \inh{fibs n ! n}, tem custo $O(1)$). Portanto:

  \[ T(\inh{n}) = f_0(\inh{n}) \]
onde $f_0$ � a fun��o de complexidade assint�tica da fun��o de cria��o
do arranjo \inh{arrFib}. Temos: 

  \[ f_0(\inh{n}) = \sum_{\inh{i}=2}^{\inh{n}} (c_0(\inh{i}) + c(\inh{i})) \]
onde $c_0(i)$ � o custo de criar a lista e $c(\inh{i})$ o custo da
contena��o dessa lista (i.e.~uso de \inh{(++)}). Temos:

   \[ c_0(\inh{i}) = \Theta(1) \]
uma vez que a criac�o envolve apenas duas opera��es de indexa��o
(\inh{arrFib!(i-2)} e \inh{arrFib!(i-1)}) e uma soma
(\inh{arrFib!(i-2) + arrFib!(i-1)}), e:

  \[ c(\inh{i}) \asymp n \]
uma vez que a complexidade assint�tica da contena��o � linear no
tamanho (\inh{i}) da primeira lista a ser concatenada (primeiro
argumento de \inh{(++)}).  Logo:

   \[ T(n) \asymp n^2 \] 

Logo, a opera��o que determina a complexidade assint�tica do programa
\inh{fib1} � a concatena��o iterativa de listas, realizada para
cria��o do arranjo \inh{arrFib}.

No caso de \inh{fib}, a complexidade assint�tica � claramente linear:

   \[ T(\inh{n}) = s(\inh{n-k}) = s(\inh{n-k-1}) + f(\inh{n}) \]
onde $s(\inh{n-k})$ � a complexidade assint�tica de \inh{step}, e
$f(n) = \Theta(1)$ � o custo de realizar opera��es de adi��o de
inteiros (\inh{k+1} e \inh{fibk_2+fibk_1}) e compara��o entre inteiros
(\inh{n==k}).

H� um dr�stico contraste entre as complexidades de tempo de \inh{fib}
e \inh{fibI}: como vemos a seguir, a complexidade de tempo de
\inh{fibI} � exponencial.

O m�todo de substituir-para-generalizar n�o fornece solu��o para a
rela��o de recorr�ncia do nosso problema. Podemos obter uma solu��o
usando uma estimativa para a complexidade de tempo igual a
$k^{\inh{n}}$. Considerando $F(n) = k^n$, temos: $k^n = k^{n-1} +
k^{n-2}$; dividindo ambos os lados por $k^{n-2}$ obtemos: $k^2 = k +
1$.  A solu��o da equa��o de segundo grau $k^2 - k - 1 = 0$ nos
fornece $k = \frac{1 \pm \sqrt{5}}{2}$. 

As duas ra�zes da equa��o de segundo grau s�o: 

  \[ \begin{array}{l}
      r_1 = \frac{1 + \sqrt{5}}{2} \\
      r_2 = \frac{1 - \sqrt{5}}{2}
    \end{array}
  \]

\ldots \ldots

  \[ F(\inh{n}) = \alpha r_1^{\inh{n}} + \beta r_2^{\inh{n}} \]
Usando as condi��es iniciais $F(0) = 0$ e $F(1) = 1$, obtemos:

  \[ \begin{array}{l}
         \alpha + \beta = 0\\
         \alpha r_1 + \beta r_2 = 1
     \end{array}
  \]
e portanto $\alpha = \frac{1}{\sqrt{5}}$ e $\beta = -\frac{1}{\sqrt{5}}$. 
Portanto: 

  \[ F(\inh{n}) = \frac{1}{\sqrt{5}} (\phi^{\inh{n}} - (1 - \phi)^{\inh{n}}) \]
onde $\phi = \frac{1 + \sqrt{5}}{2}$. 

Podemos essa usar f�rmula para obter diretamente o \inh{n}-�simo
n�mero da sequ�ncia de Fibonacci, com o cuidado de verificar a
implementa��o para evitar introdu��o de erros na manipula��o de
n�meros de ponto flutuante (� de fato surpreendente que as opera��es
de n�meros de ponto flutuante do lado direito da equa��o de
$F\inh{n})$, que incluem exponencia��es diversas de n�meros
irracionais, forne�am como resultado os n�meros da sequ�ncia de
Fibonacci). A complexidade de tempo obtida com o uso da f�rmula acima
� igual � complexidade de tempo da fun��o de exponencia��o, para a
qual existem algoritmos com ordem de complexidade logar�tmica ($O(lg
n)$).

Note tamb�m que $(1-\phi)^{\inh{n}}$ fica infinitamente pequeno quando
\inh{n} tende para infinito. Pode ser mostrado (....ref....?)  que o
resultado de $\frac{1}{\sqrt{5}} (\phi^{\inh{n}} +
\frac{1}{\phi}^{\inh{n}})$ � o mesmo de $\frac{1}{\sqrt{5}}
(\phi^{\inh{n}} + \frac{1}{\phi}^{\inh{n}})$ � o mesmo de
$\frac{1}{\phi}^{\inh{n}}$ arredondado para o inteiro mais pr�ximo.

\subsection{Vers�o Imperativa}

Uma vers�o eficiente que armazena a sequ�ncia de n�meros de Fibonacci
em um arranjo � mostrada na vers�o seguinte (usando fun��o n�o
inclu�da \ina{alocaArranjo}, para alocar arranjo, dado o n�mero de
elementos):

\begin{center}
\begin{tabular}{l}
\begin{alg}{fibA}{\decremento}
fibA(n)
  F = alocaArranjo (n)
  F(0) = 0
  F(1) = 1
  for i = 2 to n do
    F[i] = F[i-1] + F[i-2]
  return F
\end{alg}
\end{tabular}
\end{center}

A fun��o \ina{fibA} � claramente linear: ela realiza \ina{n-1}
adi��es, uma a cada itera��o. O uso de um arranjo pode ser evitado, se
apenas o \ina{n}-�simo n�mero da sequ�ncia de Fibonacci �
desejado. Nesse caso, apenas dois valores s�o necess�rios (como vimos
na vers�o funcional mostrada na subsec�o anterior).

Como mostrado na subse��o anterior, podemos usar a f�rmula:

  \[ F(\inh{n}) = (1/\sqrt{5}) (\phi^n + \frac{1}{\phi^n}) \]
onde $\phi = (1 + \sqrt{5})/2$, para obter uma complexidade de
tempo igual � complexidade de tempo da fun��o de exponencia��o.




\section{Exerc�cios}

\begin{enumerate}

\item Escreva uma fun��o com argumentos $x$ e $n$, sendo $n$ um
  inteiro positivo, e retorne o valor de $x^n$ ($x$ elevado �
  $n$-�sima pot�ncia) de modo que a complexidade de tempo da fun��o
  seja logar�tmica ($O(lg n)$).

Sua fun��o deve levar em conta que, se $n$ � par, ent�o $x^n$ � igual
ao quadrado de $x^{n/2}$ e, se $n$ � �mpar, ent�o $x^n$ � igual ao
produto de $x$ por $x^{n-1}$.

Escreva duas vers�es diferentes, uma que usa recurs�o e outra que usa
comando de repeti��o.

Prove que as duas defini��es t�m complexidade de tempo $O(lg n)$.

\end{enumerate}



%!TEX encoding = ISO-8859-1
\chapter{\colorbox{cyan}{Estruturas de Dados B�sicas}}
\label{estruturas-de-dados-basicas}

Este cap�tulo aborda listas e �rvores, suas representa��es em um computador e
opera��es b�sicas sobre essas estruturas de dados.

\section{Listas}
\label{listas}

Uma lista � uma estrutura de dados comumente usada em computa��o e pode ser definida recursivamente como a seguir.  Uma lista de elementos de
determinado tipo \inh{t} � ou i) {\em vazia} ou ii) constitu�da de um elemento de tipo \inh{t} e um uma lista de elementos do mesmo tipo \inh{t} (denominada {\em cauda} ou resto da lista).

Em uma linguagem como Haskell, que prov� suporte � defini��o de tipos
de dados recursivos, o tipo lista pode ser definido como:

\begin{center}
\begin{tabular}{l}
\begin{hask}{List,Nil,Cons}
data List a = Nil | Cons a (List a)
\end{hask}
\end{tabular}
\end{center}

O tipo de dado \inh{\List a} � um tipo recursivo, sendo \inh{Nil} e \inh{Cons} os construtores de valores desse tipo. Al�m disso, \inh{\List a} � um tipo {\em polim�rfico}: a vari�vel de tipo \inh{a} pode ser instanciada para um tipo \inh{t} qualquer, permitindo assim a defini��o de listas com elementos de desse tipo \inh{t}. Por exemplo, \inh{(Cons 1 (Cons 2 (Cons 3 Nil)))} � uma lista de tipo \inh{List Int} e \inh{(Cons True (Cons False Nil))} tem tipo \inh{List Bool}. 

A linguagem Haskell prov� uma nota��o especial para a cria��o de valores de tipo lista: \inh{[]}  � usado em lugar de \inh{Nil}, e o construtor infixado \inh{(:)}, associativo � direita, em lugar de \inh{Cons}. Por exemplo, a lista \inh{Cons 1 (Cons 2 (Cons 3 Nil))} seria representada como \inh{(1:2:3:[]}. 
Al�m disso, uma lista pode ser representada simplesmente escrevendo-se os seus elementos da lista entre colchetes, separados por v�rgulas. Ou seja, a lista \inh{\inh{(1:2:3:[]}} pode ser escrita, mais simplesmente, como \inh{[1,2,3]}. 

Em linguagens como \C, uma lista pode ser representada por meio de {\em registros} (tamb�m chamados de  "estruturas", em \C) e {\em ponteiros\/} (ou apontadores), tal como ilustrado no exemplo a seguir, que define uma lista de elementos de tipo \ina{int}. Linguagens como \C, que n�o prov�m suporte para polimorfismo, requerem a defini��o de tipos lista distintos para cada tipo particular de elementos. 

\begin{center}
\begin{tabular}{l}
\begin{alg}{ListaDeInteiros}
struct ListaDeInteiros
   int elem; 
   struct ListaDeInteiros *r;
\end{alg}
\end{tabular}
\end{center}

Uma defini��o de registro, em \C, � introduzida pela palavra-chave \ina{struct}, seguida do nome do registro -- neste caso, \ina{ListaDeInteiros}. O registro possui dois campos: um campo de tipo \ina{int} e nome \ina{elem}, e um campo de nome \ina{r}, que � um ponteiro para valores do pr�prio tipo \ina{ListaDeInteiros}. 
 
A manipula��o de valores de tipo lista em \C\ � bem mais trabalhosa. A falta de suporte para defini��o e uso de tipos recursivos e polim�rficos torna a programa��o mais dif�cil e demorada e o c�digo menos leg�vel e mais sujeito a repeti��es e a ocorr�ncias de erros. Por exemplo, para criar um valor de tipo \ina{ListaDeInteiros}, com os elementos \inh{1,2,3},  � necess�rio c�digo como o seguinte:
\index{\ina{ListaDeInteiros}}

\begin{center}
\begin{tabular}{l}
\begin{alg}{ListaDeInteiros}
struct ListaDeInteiros *p =
   malloc(sizeof(struct ListaDeInteiros))
   p->elem = 1;       p->r = malloc(sizeof(struct ListaDeInteiros))
   p->r->elem = 2;    p->r->r = malloc(sizeof(struct ListaDeInteiros))
   p->r->r->elem = 3; p->r->r->r = NULL
\end{alg}
\end{tabular}
\end{center}

Em \Haskell, o acesso a um valor \inh{v}, em uma lista \inh{x}, requer acesso a
todos os elementos anteriores a \inh{v} em \inh{x}. De fato, a representa��o interna de listas definidas em \Haskell � feita por meio de ponteiros, mas a manipula��o de ponteiros � gerada automaticamente pelo compilador da linguagem, de acordo com o c�digo do programa, em vez de ser feita diretamente pelo programador.

Uma maneira alternativa de representar listas � por meio de {\em arranjos}, 
especialmente em uma linguagem (como \C) que n�o prov� suporte a
manipula��o de valores de estruturas de dados recursivas. Utilizando essa forma de representa��o, a lista fica limitada a um n�mero m�ximo de elementos, j� que a defi��o de uma arranjo requer que o n�mero de elementos do mesmo seja especificado a priori.

\subsection{Pesquisa}
\label{pesquisa-em-lista}

Em computa��o, {\em pesquisar\/} em geral significa determinar se um
dado elemento est� presente ou n�o em uma estrutura de dados. As
subse��es seguintes tratam de opera��es de pesquisa, inser��o e
remo��o de elementos de listas, de acordo com a forma com que listas
s�o representadas.

Em ambos os casos apresentados abaixo, a pesquisa em uma lista de $n$
elementos tem complexidade $O(n)$ no pior caso, pois envolve, possivelmente, compara��o com cada elemento da lista.

\subsubsection{Vers�o funcional}

A fun��o \inh{elem}, que determina se um dado valor � elemento de uma lista dada, pode ser definida como a seguir:
\index{\inh{elem}}

\begin{center}
\begin{tabular}{l}
\begin{hask}{elem}
elem :: Eq a => a -> a -> Bool
elem a []    = False
elem a (b:x) = (a == b) || elem a x
\end{hask}
\end{tabular}
\end{center}

O tipo de \inh{elem}, em \Haskell, � um tipo {\em polim�rfico restrito\/}: a restri��o ({\em constraint\/}) (\inh{Eq a}) indica que a vari�vel de tipo \inh{a}
n�o pode ser instanciada para qualquer tipo, mas apenas para um tipo
que � membro da classe de tipos \inh{Eq}, ou seja, no caso, apenas para um
tipo para o qual exista definida uma opera��o de igualdade \inh{(==)}, para
valores desse tipo. � um erro de tipo chamar a fun��o \inh{elem} com um argumento de um tipo para o qual n�o � definida compara��o de igualdade.

\HRule
{\em Nota\/}: 

A fun��o \inh{elem}, de fato, faz parte do m�dulo \Prelude, importado
automaticamente por todos os m�dulos de programas \Haskell, sem necessidade de comando ou cl�usula expl�cita de importa��o. A defini��o de \inh{elem} no \Prelude\ � diferente da apresentada acima, e usa outras fun��es tamb�m definidas no \Prelude, como \inh{mmap} e \inh{foldr}, que s�o ferramentas importantes para defini��o de outras fun��es em Haskell. A defini��o de \inh{elem} contida no \Prelude\ � apresentada a seguir:

\begin{center}
\begin{tabular}{l}
\begin{hask}{foldr,mmap,and,or,any,all,elem}
foldr  :: (a -> b -> b) -> b -> [a] -> b
foldr f z []    =  z
foldr f z (a:x) =  f a (foldr f z x) 

mmap :: (a -> b) -> [a] -> [b]
mmap �{\tt \_}�  []   = []
mmap f (a:x) = f a : mmap f x 

and, or :: [Bool] -> Bool
and =  foldr (&&) True
or   =  foldr (||) False 

any, all :: (a -> Bool) -> [a] -> Bool
any p =  or . map p
all p =  and . map p 

elem :: (Eq a) => a -> [a] -> Bool
elem a = any (== a)
\end{hask}
\end{tabular}
\end{center}

\HRule

\subsubsection{Vers�o imperativa}

A vers�o imperativa de \ina{elem} definida a seguir recebe como argumento o valor a ser pesquisado, denotado pelo par�metro \ina{a}, juntamente com um apontador para uma lista (de tipo \ina{ListaDeInteiros}), denotado pelo par�metro \ina{l}. O algoritmo retorna retorna um apontador para o elemento da lista \ina{l} que � igual a \ina{a}, caso o argumento esteja presente na lista, e retorna \ina{NULL} em caso contr�rio.
\index{\ina{elem}}

\begin{center}
\begin{tabular}{l}
\begin{alg}{elem}
elem (a, l) 
   while (l != NULL && l->elem != a) l = l->r
   return l
\end{alg}
\end{tabular}
\end{center}

\subsection{Inser��o}
\label{insercao-em-lista}

Inserir um elemento no in�cio de uma lista � uma opera��o de
complexidade $O(1)$.

\subsubsection{Vers�o funcional}

A inser��o de um elemento no in�cio da lista � feita simplesmente, por meio do construtor de lista \inh{(:)}, ou seja:
\index{\inh{insert}}

\begin{center}
\begin{hask}{insert}
insert = (:)
\end{hask}
\end{center}

\subsubsection{Vers�o imperativa}

Na vers�o imperativa, � alocado um novo registro, de tipo \ina{ListaDeInteiros},
para armazenar o novo valor a ser inserido na lista, sendo retornada a lista resultante dessa opera��o.
\index{\ina{insert}}

\begin{center}
\begin{tabular}{l}
\begin{alg}{insert}
insert (a, l)
   p = malloc(sizeof(struct ListaDeInteiros))
   p->elem = a
   p->r = l
   return p
\end{alg}
\end{tabular}
\end{center}

\subsection{Remo��o}
\label{remocao-de-lista}

Remover um elemento de uma lista � uma opera��o de complexidade $O(n)$
no pior caso, pois � necess�rio procurar o elemento a ser removido.

A Se��o \ref{lista-duplamente-encadeada} redefine o tipo de lista
encadeada para uma vers�o de listas duplamente encadeadas, e reescreve
as fun��es \ina{elem} e \ina{insert}, para definir \ina{delete} por meio de uma
chamada � fun��o \ina{elem}, seguida de chamada a \ina{insert}.

\subsubsection{Vers�o funcional}

A vers�o funcional cria uma nova lista, que n�o tem o elemento
passado como par�metro:
\index{\inh{delete}}

\begin{center}
\begin{tabular}{l}
\begin{hask}{delete}
delete :: Eq a => a -> [a] -> [a]
delete �{\tt \_}� [] = []
delete a (b:x)
   | a == b    = x
   | otherwise = b:delete a x
\end{hask}
\end{tabular}
\end{center}

\subsubsection{Vers�o imperativa}

A vers�o imperativa de \ina{delete}, apresentada a seguir, n�o cria uma nova lista: usa um ponteiro -- \ina{prev} -- para percorrer a lista at� encontrar o elemento a ser removido e, quando este � encontrado, modifica a estrutura de encadeamento da lista, removendo este elemento. 

\begin{center}
\begin{tabular}{l}
\begin{alg}{delete}
delete (a, l) 
   struct ListaDeInteiros *prev = NULL;  
   *p = l
   while (p != NULL && p->elem != a)
      prev = p
      p = p->r
      if (prev == NULL) 
         return l->r
      else { prev->r = p->r; return l }
\end{alg}
\end{tabular}
\end{center}
             

\subsection{Lista duplamente encadeada}
\label{lista-duplamente-encadeada}

\ldots \ldots .....

\subsection{Pilha}
\label{pilha}

Uma estrutura de dados \ina{Pilha} caracteriza-se pelo fato de que as opera��es de inser��o, acesso e remo��o de elementos s�o feitas em apenas
um de seus lados (ou extremidades). Essa pol�tica de uso � comumente  chamada LIFO (do ingl�s, {\em last-in first-out\/}: o �ltimo a ser inserido � o primeiro a ser removido da pilha. A implementa��o de uma \ina{Pilha} inclui as  com as opera��es: i) criar pilha vazia, ii) empilhar elemento, iii) desempilhar elemento, iv) obter elemento do topo da pilha, e v) testar se pilha est� vazia. 

\subsubsection{Vers�o funcional}
\index{\inh{Pilha}}
Em \Haskell, a implementa��o de \inh{Pilha} � obtida diretamente das opera��es definidas sobre listas, isto �:

\begin{center}
\begin{tabular}{l}
\begin{hask}{vazia,empilhar,desempilhar,topo,estaVazia}
vazia = []
empilhar = (:)
desempilhar (�{\tt \_}�:x) = x
topo (a:�{\tt \_}�) = a
estaVazia = null
\end{hask}
\end{tabular}
\end{center}
                  
Em geral, vamos procurar simplificar o c�digo de nossos programas,
n�o tratando casos de erro, na maioria das vezes, por motivos did�ticos. Entretanto, em programas completos n�o podemos esquecer de tratar todos os casos poss�veis para os dados de entrada. Um m�dulo em Haskell que trata
todos esses casos poss�veis para os dados de entrada das opera��es
acima � mostrado na Figura \ref{fig-Pilha}.

O m�dulo \inh{Pilha} implementa o que � chamado em computa��o de um {\em tipo abstrato de dados\/}, que � um tipo com opera��es de cria��o,
modifica��o e consulta sobre valores desse tipo. Por exemplo, \inh{vazia} � uma opera��o de cria��o (nesse caso, a �nica);
\inh{empilhar} e \inh{desempilhar} s�o opera��es de modifica��o; \inh{topo} e \inh{estaVazia} s�o opera��es de consulta. 

Em uma defini��o de um tipo abstrato de dados, a defini��o de tipo e das opera��es (para cria��o, modifica��o e consulta) sobre valores
do tipo devem ser contidas em um mesmo trecho de programa (em geral,
um m�dulo), e a representa��o usada n�o � ``vis�vel'' para quem usa valores do tipo. Ou seja, um tipo abstrato � constitu�do de um tipo, munido de um conjunto de opera��es sobre valores desse tipo. Qualquer outra opera��o sobre valores do tipo apenas pode ser implementada por meio dessas opera��es previamente definidas.

Para definir um tipo abstrato \inh{Pilha}, em \Haskell, usamos uma defini��o de um novo tipo, introduzida pela palavra-chave \inh{newtype}. O tipo \inh{Pilha} possui um �nico construtor de valores \inh{mkPilha} (de mesmo nome do construtor de tipo). Esse mecanismo � usado para ocultar a representa��o do tipo abstrato: o construtor de  que n�o � exportado pelo m�dulo em que o tipo � definido: o construtor de valores \inh{mkPilha} n�o � exportado, apenas o construtor de tipos \inh{Pilha}. Para melhor legibilidade, definimos tamb�m o tipo de cada uma das fun��es. Veja Figura \ref{fig-Pilha}.
\end{document}

\begin{figure}

\begin{center}
\begin{tabular}
\begin{hask}{Pilha,vazia,empilhar,desempilhar,topo,estaVazia}
module Pilha (Pilha, vazia, empilhar, desempilhar, topo, estaVazia) where
   newtype Pilha a = mkPilha [a] 

vazia :: Pilha $a
vazia = mkPilha []

empilhar :: a -> Pilha a -> Pilha a
empilhar e (mkPilha p) = Pilha (e:p)

desempilhar :: Pilha a -> Pilha a
desempilhar (mkPilha []) = error "Fun��o desempilhar chamada com pilha vazia"
desempilhar (mkPilha (�{\tt \_}�:p)) = mkPilha p

topo:: Pilha a -> a
topo (mkPilha [])   = error "Fun��o topo chamada com pilha vazia"
topo (mkPilha (e:�{\tt \_}�)) = e 

estaVazia:: Pilha a -> Bool
estaVazia (mkPilha $p$) = null p
\end{hask}
\end{tabular}
\end{center}

\label{fig-Pilha}
\caption{Tipo abstrato \Pilha\ em Haskell}
\end{figure}



\subsubsection{Vers�o imperativa}

Considerando \pilha\ como um registro com campos \topo\ e \elems,
sendo \elems\ um arranjo de $n$ elementos --- indexado de $0$ a {\tt
  $n$-1} --- e \topo\ uma vari�vel inteira que indica o �ndice do
�ltimo elemento inserido, as opera��es em uma pilha podem ser
implementadas como a seguir (desconsiderando casos de erro:
desempilhar de uma pilha vazia e empilhar em uma pilha cheia).

\newcommand{\nome}{{\it nome\/}}

O comando \with\ serve para tornar os nomes de campos de um registro
vis�veis: \with\ $r$ evita que se tenha que prefixar os nomes dos
campos do registro $r$ com o registro (\nome\ pode ser usado em vez de
{\tt $r$.\nome}).

\progb{
  \vazia\ (\pilha) \{ \pilha.\topo\ = -1 \}\\ \ \hspace*{.2cm} \\
  \empilhar\ (\eelem, \pilha) \\
      \hspace*{.2cm} \with\ \pilha\  \\
        \hspace*{1cm} \topo\ $\leftarrow$ \topo\ + 1\\
        \hspace*{1cm} \elems[\topo] $\leftarrow$ \eelem\\ \ \hspace*{.2cm} \\
  \desempilhar\ (\pilha) \\ 
      \hspace*{.2cm} \with\ \pilha\ \{ \topo\ $\leftarrow$ \topo\ - 1 \} \\ \ \hspace*{.2cm} \\
  \topo\ (\pilha) \{ \with\ \pilha\ \{ \return\ \elems[\topo] \} \}\\ \ \hspace*{.2cm} \\ 
  \estaVazia\ (\pilha) \{ \return\ \pilha.\topo\ == -1 \}
 }

\subsection{Fila}
\label{fila}

Em uma {\em fila} a inser��o � feita de um lado e a remo��o � feita do
outro lado da estrutura de dados. Isso implica em uma pol�tica algumas
vezes chamada de FIFO ({\em first-in first-out\/}: o primeiro a ser
inserido � o primeiro a ser removido da fila.

Uma fila, com opera��es de i) criar fila vazia, ii) entrar (inserir
elemento) na fila, iii) sair (tirar elemento) da fila, iv) obter
elemento do in�cio da fila, e v) testar se fila est� vazia, pode ser
implementada como a seguir.

\subsubsection{Vers�o funcional}

\newcommand{\frente}{{\it frente\/}}
\newcommand{\tras}{{\it tr�s\/}}
\newcommand{\fila}{{\it fila\/}}

N�o � eficiente fazer acesso ao �ltimo elemento de uma lista em
Haskell. A implementa��o padr�o de filas por meio de listas usa assim
duas listas, \frente\ e \tras. Elementos entram na lista \tras\ e saem
na lista \frente.

A fun��o \fila\ � usada para garantir o invariante de que se
\frente\ est� vazia, ent�o \tras\ est� vazia (e portanto a fila est�
vazia).

\newcommand{\reverse}{{\it reverse\/}}
\newcommand{\sair}{{\it sair\/}}
\newcommand{\entrar}{{\it entrar\/}}
\newcommand{\inicio}{{\it in�cio\/}}

\progb{\vazia\ = ([],[])\\ \ \hspace*{.2cm} \\
      \entrar\ $e$ (\frente,\tras) = \fila\ (\frente, $e$ : \tras)\\ \ \hspace*{.2cm} \\
      \fila\ ([], \tras) = (\reverse\ \tras, [])\\
      \fila\ $f$        = $f$\\ \ \hspace*{.2cm} \\
      \sair\ ($e$:\frente, \tras) = (\frente,\tras)\\ \ \hspace*{.2cm} \\
      \estaVazia\ (\frente,\_) = \null\ \frente\\ \ \hspace*{.2cm} \\
      \inicio\ ($e$:\frente,\_) = $e$
     }

\subsubsection{Vers�o imperativa}
\label{fila-imperativa}

\newcommand{\fim}{{\it fim\/}}

Considerando \fila\ como um registro com campos \inicio, \fim\ e
\elems, sendo \elems\ um arranjo de $n$ elementos --- indexado de $0$
a {\tt $n$-1}.  Os �ndices do primeiro e do �ltimo elementos inseridos
s�o armazenados respectivamente nas vari�veis \inicio\ e \fim. 

A fila est� vazia quando {\tt \inicio\ == \fim}. A fila est� cheia
quando {\tt \inicio\ == \fim\ + 1}, isto �, a fila � circular: o
�ndice {\tt 0} segue o �ndice {\tt $n$-1}. 

As opera��es em uma fila podem ser implementadas como a seguir,
desconsiderando casos de erro: sair de uma fila vazia e entrar em uma
fila cheia. O operador {\tt \%} � usado para retornar o resto da
divis�o do primeiro operando pelo segundo.

\progb{
  \vazia\ (\fila) \{ \with\ \fila\ \{ \inicio\ = \fim\ = 0 \} \} \\ \ \hspace*{.2cm} \\
  \entrar\ (\eelem, \fila) \\
      \hspace*{.2cm} \with\ \fila\  \\
          \hspace*{1cm} \elems[\fim] $\leftarrow$ \eelem\\
          \hspace*{1cm} \fim\ $\leftarrow$ (\fim\ + 1) \% $n$\\ \ \hspace*{.2cm} \\
  \sair\ (\fila) \\
      \hspace*{.2cm} \with\ \fila\ \{ \inicio\ $\leftarrow$ (\inicio\ + 1) \% $n$ \}\\ \ \hspace*{.2cm} \\
  \inicio\ (\fila) \{ \with\ \fila\ \{ \return\ \elems[\inicio] \} \}\\ \ \hspace*{.2cm} \\
  \estaVazia\ (\pilha) \{ \with\ \fila\ \{ \return\ \inicio\ == \fim\ \} \} 
 }

%!TEX encoding = ISO-8859-1
\section{�rvores}
\label{sec:arvores}
\index{arvores}

Uma �rvore pode ser vista como uma estrutura de dados recursiva, definida como sendo ou i) vazia (uma folha) ou ii) um nodo contendo um elemento e um certo n�mero de ramos (ou nodos), que cont�m sub-�rvores:

\begin{center}
\begin{tabular}{l}
\begin{hask}{Tree,Leaf,Node}
data Tree a = Leaf | Node a [Tree a]
\end{hask}
\end{tabular}
\end{center}

Na defini��o acima, o construtor \inh{Leaf} constr�i uma �rvore vazia, ou folha, e o construtor \inh{Node} constr�i uma �rvore contendo uma informa��o de tipo \inh{a} e um lista de sub�rvores.

Outra poss�vel defini��o considera que a informa��o � armazenada nas folhas, em vez de nos nodos internos:

\begin{center}
\begin{tabular}{l}
\begin{hask}{Tree,Leaf,Node}
data Tree' a = Leaf' a | Node' [Tree' a]
\end{hask}
\end{tabular}
\end{center}

Uma �rvore com exatamente duas sub-�rvores (possivelmente vazias) � chamada de �rvore bin�ria e pode ser definida como:

\begin{center}
\begin{tabular}{l}
\begin{hask}{BTree,BLeaf,BNode}
 data BTree a = BLeaf a | BNode (BTree a) (BTree a)
\end{hask}
\end{tabular}
\end{center}

%---- acho que esse par�grafo deve ser eliminado
%---- � prematuro falar em grafo e nem � apresentado um desenho
%---- talvez isso possa ser deixado como nota
%
%Uma �rvore pode tamb�m ser definida como um {\em grafo} conexo e  ac�clico. Um {\em grafo\/} � simplesmente um conjunto de v�rtices e de arestas entre esses v�rtices. Dizemos que dois v�rtices $a$ e $b$ do grafo s�o {\em adjacentes\/} se est�o conectados por uma aresta -- usualmente representada como um par$(a,b)$. Um grafo � {\em conexo\/} se todo v�rice � adjacente a algum outro. Um {\em caminho\/} de um v�rtice $a$ a um v�rtice $b$ � uma sequ�ncia de v�rtices adjacentes, tendo $a$ como primeiro e $b$ como �ltimo v�rtice. Um {\em ciclo\/} � um caminho que inicia e termina no mesmo v�rtice, n�o repetindo nenhuma aresta. Um grafo � {\em ac�clico\/} se n�o cont�m nenhum ciclo. Um grafo ac�clico, mas n�o for conexo, � uma floresta, isto �, um conjunto de �rvores.
%
%-----------------------------------------------------------

Em linguagens que prov�em suporte ao uso de ponteiros, mas n�o �
defini��o e manipula��o direta de tipos recursivos, a representa��o de
�rvores bin�rias pode ser feita com o uso de ponteiro como mostra o
exemplo a seguir:

\begin{center}
\begin{tabular}{l}
\begin{alg}{ArvoreBinariaDeInteiros}
struct ArvoreBinariaDeInteiros
     int elem
     struct ArvoreBinariaDeInteiros *esq, dir 
\end{alg}
\end{tabular}
\end{center}

Os campos \ina{esq} e \ina{dir} de um nodo s�o ponteiros, possivelmente nulos,
para sub-�rvores.

Para �rvores n�o bin�rias, pode ser usada a seguinte representa��o, que
podemos chamar de {\em representa��o com primog�nito-irm�o-e-pai}:

\begin{center}
\begin{tabular}{l}
\begin{alg}{ArvoreDeInteiros}
struct ArvoreDeInteiros
   int elem;
   struct ArvoreDeInteiros *primogenito, irmao, pai 
\end{alg}
\end{tabular}
\end{center}

A representa��o da �rvore da Figura \ref{fig:Arv1} � mostrada na Figura \ref{fig:Rep-arv1}.

\begin{figure}

xxxx
\caption{�rvore exemplo}
\label{fig:Arv1}
\end{figure}

\begin{figure}

yyyyy
\caption{Representa��o da �rvore exemplo por valor do tipo \ina{ArvoreDeInteiros}}
\label{fig:Rep-arv1}
\end{figure}
\end{document}


>>>>>>> FETCH_HEAD

%% !TEX encoding = ISO-8859-1
\chapter{Algoritmos e �rvores de Pesquisa}
\label{algoritmos-de-pesquisa}

Pesquisar em computa��o significa encontrar um dado valor, chamado de
{\em chave da pesquisa\/}, dentre v�rios valores existentes. Os
valores existentes podem estar representados de v�rias formas, mas
vamos tratar neste livro apenas de listas e �rvores.  Mesmo nos
restringindo apenas a essas formas de representa��o de valores,
existem v�rios algoritmos de pesquisa. 

Na se��o \ref{sec:pesquisa-em-lista} apresentamos um algoritmo simples
de {\em pesquisa sequencial\/} em listas (incluindo representa��o com
arranjos). Duas varia��es simples dessa pesquisa sequencial s�o
apresentadas nos exerc�cios resolvidos. A primeira � baseada em
pesquisa em lista ordenada, que termina a pesquisa sequencial quando a
chave da pesquisa � encontrada ou quando se torna maior do que um
elemento da lista (supondo ordem crescente dos valores na lista). A
segunda usa o que � chamado de {\em sentinela} --- um elemento
adicionado ao extremo (tipicamente, de arranjo), para evitar teste
para verificar chegada a esse extremo (por isso, � usada somente
quando o n�mero de elementos que pode ser armazenado � limitado, como
ocorre no caso de arranjos).

A se��o \ref{pesquisa-binaria} apresentamos o eficiente algoritmo de
pesquisa em �rvore bin�ria, chamada de {\em pesquisa bin�ria}. As
se��es seguintes apresentam varia��es da pesquisa bin�ria, que usam
opera��es para balanceamento da �rvore na qual a pesquisa � feita, com
o objetivo de aumentar a efici�ncia da pesquisa.

% !TEX encoding = ISO-8859-1
\section{Pesquisa binária}
\label{sec:pesquisa-binaria}

.... 




\section{Exerc�cios Resolvidos}

\begin{enumerate}

\item .... pesquisa em lista ordenada ....

\item .... pesquisa em arranjo com sentinela ....

\end{enumerate}

\section{Exerc�cios}




%\chapter{{Algoritmos de Ordena��o}
\label{algoritmos-de-ordenacao}

Uma importante metodologia para constru��o de algoritmos � chamada de
{\em dividir-para-conquistar}. Ela consiste em quebrar um problema em
dois ou mais sub-problemas, at� que se obtenha um sub-problema que
pode ser resolvido diretamente.  As solu��es dos sub-problemas s�o
ent�o combinadas para obten��o de uma solu��o para o problema
original. A t�cnica � usada por exemplo no caso de algoritmos
eficientes de ordena��o como quicksort e merge-sort. A corre��o de
algoritmos obtidos pelo uso do m�todo envolve indu��o matem�tica, e a
an�lise de efici�ncia � usualmente determinada pela solu��o de
rela��es de recorr�ncia.


%% !TEX encoding = ISO-8859-1
\chapter{Algoritmos de Dispers�o (Hash)}
\label{dispersao}
\index{dispersao@dispers�o}
\index{dispersao@dispers�o!algoritmos de}
\index{hash}

Este cap�tulo apresenta uma t�cnica simples e eficiente de
implementa��o de dicion�rios que ilustra bem o compromisso entre tempo
e espa�o no projeto de implementa��o de algoritmos e estruturas de
dados. O uso de mais espa�o pode diminuir o tempo e o uso de menos
espa�o pode aumentar o tempo de execu��o, mas h� sempre um limite
superior para a quantidade de espa�o a ser utilizada.

\index{dicionario@dicion�rio} Em computa��o, um dicion�rio � um tipo
abstrato que define opera��es de inser��o, remo��o e pesquisa.

Os dados s�o usualmente organizados em registros com diversos campos,
dentre eles um ou mais campos que s�o usados como {\em chave\/} para a
pesquisa, inser��o ou remo��o.

\index{fun��o de dispers�o} \index{�ndice de dispers�o} Algoritmos de
dispers�o (em ingl�s, {\em hashing\/}) se baseiam na ideia simples de
distribuir os dados em um vetor de certo tamanho --- que vamos chamar
de tabela de dispers�o --- usando uma {\em fun��o de dispers�o\/}, que
associa um �ndice do vetor a cada chave.

\index{\disperse} Por exemplo, se a chave � um inteiro positivo ou
nulo e a tabela de dispers�o tem tamanho $m$, a fun��o de dispers�o
pode ser a fun��o $\disperse$ tal que $\disperse(v) = v \% m$, onde
$\%$ � a fun��o que calcula o resto da divis�o de n�meros inteiros.

% Se a chave for um caractere, a fun��o de dispers�o $\disperse$ pode
% ser tal que $\delta(v) = \ord(v) \% m$, onde $\ord$ � a fun��o que
% associa a cada caractere o c�digo inteiro que o
% representa. Computadores usam um c�digo para representar cada
% caractere; hoje em dia usualmente Unicode ou uma varia��o do c�digo
% Unicode, antigamente usava-se c�digo Ascii.

\index{compromisso entre tempo e espa�o} A distribui��o dos valores a
serem pesquisados em uma tabela � um bom exemplo do compromisso entre
tempo e espa�o, fundamental no projeto de algoritmos. Se n�o houver
limita��o de mem�ria, poderia ser definido um tamanho de vetor
bastante grande de modo que um �nico �ndice fosse associado a cada
valor. Se n�o houver limita��o de tempo, o projeto poderia n�o usar
dispers�o (i.e.~considerar vetor de tamanho nulo) e usar algoritmos
sequenciais em uma lista. Na pr�tica, deve-se procurar determinar um
tamanho que n�o gaste espa�o de mem�ria de mais e nem de menos, para a
tabela de dispers�o. H� algoritmos que dinamicamente alteram
(tipicamente, dobram) o tamanho da tabela, quando o {\em fator de
  ocupa��o\/} --- n�mero de elementos existentes dividido pelo tamanho
(n�mero de �ndices) da tabela --- alcan�a um valor pr�-estabelecido
(igual ou pouco maior que 0,5).  \index{fator de ocupa��o!de tabela de
  dispers�o}

\index{colis�o!em algoritmos de dispers�o} Uma {\em colis�o\/} ocorre
quando a fun��o de dispers�o retorna o mesmo �ndice para duas ou mais
chaves. Colis�es devem ocorrer raramente se o tamanho da tabela de
dispers�o � grande e uma boa fun��o de dispers�o � usada. Mas um
mecanismo de tratamento de colis�es � em geral necess�rio em todo
algoritmo de dispers�o, devido � possibilidade de ocorr�ncia de
colis�es.

A dispers�o de valores se baseia no fato de que � mais eficiente
procurar um valor em um subconjunto de todos os valores, i.e.~o
subconjunto dos valores para os quais a fun��o de dispers�o fornece o
mesmo resultado. A pesquisa usando dispers�o � o m�todo conhecido mais
poderoso e o mais usado para pesquisa de dados. A maioria dos sistemas
de recupera��o de dados usados atualmente s�o baseados em dispers�o.

\index{tratamento de colis�es!em algoritmos de dispers�o!aberto}
\index{tratamento de colis�es!em algoritmos de dispers�o!fechado} H�
duas vers�es de tratamento de colis�es em algoritmos de dispers�o:
{\em aberto\/} e {\em fechado}.

\index{tratamento de colis�es!em algoritmos de dispers�o!aberto}
\index{encadeamento separado}
No tratamento aberto, tamb�m chamado de tratamento de colis�o por {\em
  encadeamento separado\/} (em ingl�s, {\em separate chaining\/}), a
tabela de dispers�o � uma tabela de apontadores para lista de
elementos para os quais houve colis�o.

\index{endere�amento aberto}
\index{sondagem linear}
\index{tratamento de colis�es!em algoritmos de dispers�o!fechado}
No tratamento fechado, tamb�m chamado de {\em endere�amento aberto\/}
(em ingl�s, {\em open addressing\/}), todas as chaves s�o armazenadas
na pr�pria tabela (o que implica que o tamanho da tabela � maior que o
n�mero de chaves inseridas). Diferentes estrat�gias podem ser usadas
para resolu��o de conflitos, mas a mais simples --- chamada de {\em
  sondagem linear\/} (em ingl�s, {\em linear probing\/}) --- usa a
primeira posi��o, seguinte � que ocorreu a colis�o, que est� vazia
(considerando a tabela como circular, isto �, a primeira posi��o da
tabela segue a �ltima). Embora pesquisa e inser��o sejam relativamente
simples de implementar segundo a t�cnica de sondagem linear, a remo��o
de chaves � mais complicada. Em geral, � usado um s�mbolo
especialmente reservado para indicar que uma chave foi removida da
posi��o. N�o vamos abordar o tratamento fechado de colis�es.




\pagebreak
\thispagestyle{empty}
\pagebreak
\pagestyle{fancyplain}

\bibliographystyle{plain}
\bibliography{livro}

\appendix
\appendix
\chapter{\Haskell}
\label{Ap-Haskell}

A linguagem \Haskell\ \ldots

\chapter{Rela��es de Recorr�ncia}
\label{relacoes-de-recorrencia}



\printindex

\end{document}
