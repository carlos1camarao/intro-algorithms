\chapter{{Algoritmos de Ordena��o}
\label{algoritmos-de-ordenacao}

Uma importante metodologia para constru��o de algoritmos � chamada de
{\em dividir-para-conquistar}. Ela consiste em quebrar um problema em
dois ou mais sub-problemas, at� que se obtenha um sub-problema que
pode ser resolvido diretamente.  As solu��es dos sub-problemas s�o
ent�o combinadas para obten��o de uma solu��o para o problema
original. A t�cnica � usada por exemplo no caso de algoritmos
eficientes de ordena��o como quicksort e merge-sort. A corre��o de
algoritmos obtidos pelo uso do m�todo envolve indu��o matem�tica, e a
an�lise de efici�ncia � usualmente determinada pela solu��o de
rela��es de recorr�ncia.
