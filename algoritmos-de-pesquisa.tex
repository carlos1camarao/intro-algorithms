\chapter{Algoritmos de Pesquisa}
\label{algoritmos-de-pesquisa}

Pesquisar em computa��o significa encontrar um dado valor dentre
v�rios valores existentes. Os valores existentes podem estar
representados como um conjunto ou qualquer forma de agrupamento de
valores. 

%% Existem v�rios algoritmos de pesquisa. Na se��o
%%-- \ref{pesquisa-em-lista} apresentamos um algoritmo simples de {\em
%%--   pesquisa sequencial}. Na se��o --
%%-- \ref{pesquisa-sequencial-com-valores-ordenados} apresentamos pequena
%%-- varia��o da pesquisa sequencial usada quando os valores que est�o
%%-- sendo pesquisados est�o ordenados. Na se��o
%%-- \ref{pesquisa-sequencial-com-sentinela} apresentamos % outra pequena
%%-- varia��o (otimiza��o) da pesquisa sequencial, que usa o que � chamado
%%-- de {\em sentinela}. Sentinela � um elemento adicionado a extremo de
%%-- estrutura de dados para evitar teste para verificar chegada a esse
%%-- extremo.

A se��o \ref{pesquisa-binaria} apresentamos o eficiente algoritmo de
{\em pesquisa bin�ria\/}. 

.......

-- \subsection{Pesquisa sequencial com valores ordenados}
-- \label{pesquisa-sequencial-com-valores-ordenados}

-- \subsection{Pesquisa sequencial com sentinela}
-- \label{pesquisa-sequencial-com-sentinela}

% !TEX encoding = ISO-8859-1
\section{Pesquisa binária}
\label{sec:pesquisa-binaria}

.... 




