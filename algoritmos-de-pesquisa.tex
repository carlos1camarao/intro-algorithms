% !TEX encoding = ISO-8859-1
\chapter{Algoritmos e �rvores de Pesquisa}
\label{algoritmos-de-pesquisa}

Pesquisar em computa��o significa encontrar um dado valor, chamado de
{\em chave da pesquisa\/}, dentre v�rios valores existentes. Os
valores existentes podem estar representados de v�rias formas, mas
vamos tratar neste livro apenas de listas e �rvores.  Mesmo nos
restringindo apenas a essas formas de representa��o de valores,
existem v�rios algoritmos de pesquisa. 

Na se��o \ref{sec:pesquisa-em-lista} apresentamos um algoritmo simples
de {\em pesquisa sequencial\/} em listas (incluindo representa��o com
arranjos). Duas varia��es simples dessa pesquisa sequencial s�o
apresentadas nos exerc�cios resolvidos. A primeira � baseada em
pesquisa em lista ordenada, que termina a pesquisa sequencial quando a
chave da pesquisa � encontrada ou quando se torna maior do que um
elemento da lista (supondo ordem crescente dos valores na lista). A
segunda usa o que � chamado de {\em sentinela} --- um elemento
adicionado ao extremo (tipicamente, de arranjo), para evitar teste
para verificar chegada a esse extremo (por isso, � usada somente
quando o n�mero de elementos que pode ser armazenado � limitado, como
ocorre no caso de arranjos).

A se��o \ref{pesquisa-binaria} apresentamos o eficiente algoritmo de
pesquisa em �rvore bin�ria, chamada de {\em pesquisa bin�ria}. As
se��es seguintes apresentam varia��es da pesquisa bin�ria, que usam
opera��es para balanceamento da �rvore na qual a pesquisa � feita, com
o objetivo de aumentar a efici�ncia da pesquisa.

% !TEX encoding = ISO-8859-1
\section{Pesquisa binária}
\label{sec:pesquisa-binaria}

.... 




\section{Exerc�cios Resolvidos}

\begin{enumerate}

\item .... pesquisa em lista ordenada ....

\item .... pesquisa em arranjo com sentinela ....

\end{enumerate}

\section{Exerc�cios}


